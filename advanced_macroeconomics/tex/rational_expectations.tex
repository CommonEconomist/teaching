\documentclass{beamer}
\usetheme{}
\usecolortheme{dolphin}           
\useinnertheme{circles}
\setbeamertemplate{itemize items}[default]
\setbeamertemplate{enumerate items}[default]
\usepackage[T1]{fontenc}
\usepackage[utf8]{inputenc}
\usepackage{lmodern}
\usepackage{amsmath}
\usepackage{booktabs} 
\usepackage{graphicx}        
\usepackage{array}
\usepackage{color}
\makeatletter
\def\zapcolorreset{\let\reset@color\relax\ignorespaces}
\def\colorrows#1{\noalign{\aftergroup\zapcolorreset#1}\ignorespaces}
\makeatother
\graphicspath{{/home/swl/Dropbox/ucd/advanced_macro/figures/}} 
\setbeamertemplate{navigation symbols}{}
\setbeamertemplate{footline}[frame number]

%--------------------------------------
\title{Rational expectations}
\author{School of Economics, University College Dublin}
\date{Spring 2018}
\begin{document}

%--------------------------------------
\begin{frame}
 \titlepage
\end{frame}
%--------------------------------------

%--------------------------------------
\begin{frame}
  \textbf{DSGE} models have both backward and forward looking elements:
  Backward looking dynamics come from identities
  \begin{align}
     K_t= (1-\gamma) K_{t-1} + I_t
   \end{align} 
  Forward looking dynamics from optimising behaviour
  \begin{itemize}
    \item What agents expect to happen tomorrow is very important for what they decide to do today.
  \end{itemize}
  Modeling this requires assumptions about how people formulate expectations
  \begin{itemize}
    \item DSGE approach relies on the idea that people have \textbf{rational expectation}
  \end{itemize}
\end{frame}
%--------------------------------------

%--------------------------------------
\begin{frame}
  \textbf{Rational expectations} entails that agents
  \begin{enumerate}
    \item Use publicly available information efficiently (i.e. no systematic mistakes)
    \item Understand the structure of the economy and base their expectation on this knowledge
  \end{enumerate}
  \medskip
  Rational expectation is a baseline assumption about people's behaviour.
  \begin{align}
    y_t=x_t + a\mathbb{E}_ty_{t+1}
  \end{align}
  Today's value of $y$ is determined by  
\begin{enumerate}
  \item $x$
  \item tomorrow's expected value of $y$
\end{enumerate}
\end{frame}
%--------------------------------------

%--------------------------------------
\begin{frame}
  Under rational expectations agents formulate their expectations that is consistent with  
\begin{align}
  \mathbb{E}_ty_{t+1} &= \mathbb{E}_tx_{t+1} + a\mathbb{E}_t\mathbb{E}_{t+1}y_{t+2}\\ \nonumber
  \mathbb{E}_ty_{t+1} &= \mathbb{E}_tx_{t+1} + a\mathbb{E}_ty_{t+2}
\end{align}
\begin{align}
  \mathbb{E}_t\mathbb{E}_{t+1}y_{t+2} = \mathbb{E}_ty_{t+2}
\end{align}
\medskip
This is the \textbf{Law of Iterated Expectations}
\begin{itemize}
  \item It is not rational for me to expect to have a different expectation next period for $y_{t+2}$ than the one that I have today.
  \item Based on the information set available at $t$.
\end{itemize}
\end{frame}
%--------------------------------------

%--------------------------------------
\begin{frame}
  Use repeated substitution to get solution for model  
\begin{align}  
    y_t = x_t + a\mathbb{E}_tx_{t+1} + a^2\mathbb{E}_ty_{t+2} 
\end{align}
Include additional periods
\begin{align}
  y_t &= x_t + a\mathbb{E}_tx_{t+1} + a^2\mathbb{E}_ty_{t+2}+ .....\\ \nonumber &+a^{N-1}\mathbb{E}_tx_{t+N-1} + a^{N}\mathbb{E}_ty_{t+N}\\ \nonumber
  y_t &= \sum^{N-1}_{k=0}a^k\mathbb{E}_tx_{t+k} + a^{N}\mathbb{E}_ty_{t+N}\\ \nonumber
  y_t &= \sum^\infty_{k=0}a^k\mathbb{E}_tx_{t+k}
\end{align}
 This assumes that 
 \begin{align}
   \lim_{N \rightarrow \infty} a^{N}\mathbb{E}_ty_{t+N} = 0
 \end{align}
\end{frame}
%--------------------------------------

%--------------------------------------
\begin{frame}
  \textbf{Asset pricing}
  Consider asset which can be bought for $P_t$, yielding dividend of $D_t$
  \begin{itemize}
    \item There is a close alternative to this asset which will yield a guaranteed rate of return $r$. 
  \end{itemize}
  A risk neutral investor will only hold the asset if the yield is the as the same rate of return
\begin{align}  
  \frac{D_t + \mathbb{E}tP_{t+1}}{P_t}&=1+r  
\end{align}
\medskip
\begin{align}
  P_t&= \frac{D_t}{1+r} + \frac{\mathbb{E}_tP_{t+1}}{1+r}\\ \nonumber
  P_t&= \sum^{\infty}_{k=0}\left(\frac{1}{1+r}\right)^{k+1}\mathbb{E}_tD_{t+k}   
\end{align}
\end{frame}
%--------------------------------------

%--------------------------------------
\begin{frame}
  \textbf{Backward solution}  
 \begin{align}
    y_t=x_t + a\mathbb{E}_ty_{t+1} \\ \nonumber
    y_t= x_t + ay_{t+1} + a\epsilon_{t+1}
 \end{align}
  $\epsilon_{t+1}$ is a forecast error that cannot be predicted at time $t$. 
  Can move time index back one period
  \begin{align}
  y_t &= x_t + ay_{t+1} + a\epsilon_{t+1}\\ \nonumber
  y_{t-1} &= x_{t-1} + ay_t + a\epsilon_t\\ \nonumber
  ay_t &= y_{t-1} - x_{t-1} - a\epsilon_t\\ \nonumber
  y_t &= a^{-1}y_{t-1} - a^{-1}x_{t-1} - \epsilon_t 
\end{align}
\begin{align}
  y_t&= -\sum^{\infty}_{k=0}a^{-k}\epsilon_{t-k}-\sum^{\infty}_{k=1}a^{-k}x_{t-k}
\end{align}
\end{frame}
%--------------------------------------

%--------------------------------------
\begin{frame}
  We have two solutions
  \begin{align}
    y_t &= \sum^\infty_{k=0}a^k\mathbb{E}_tx_{t+k} \\
    y_t&= -\sum^{\infty}_{k=0}a^{-k}\epsilon_{t-k}-\sum^{\infty}_{k=1}a^{-k}x_{t-k}
  \end{align}
  Both are correct, which one we use depends on $a$
  \begin{align}
    |a|>1
  \end{align}
  Weights on forward solution are explosive: use backward solution
  \begin{itemize}
    \item Forward solution will not converge to finite sum
    \item $y_t$ depends more on values of $x_t$ far in the distant future than on today's values
  \end{itemize}  
\end{frame}
%--------------------------------------

%--------------------------------------
\begin{frame} 
  \begin{align}
    |a|<1
  \end{align}
  Weights on backward solution are explosive and need to use forward solution
  \begin{itemize}
    \item Path of $x_t$ will tell path of $y_t$
  \end{itemize}
  Most cases assume that $|a|<1$
  \begin{align}
    \lim_{N \rightarrow \infty} a^N \mathbb{E}_t y_{t+N} = 0
  \end{align}
   $y_t$ can't grow too fast (transversality condition)
\end{frame}
%--------------------------------------

%--------------------------------------
\begin{frame}
  \textbf{Rational bubbles:} If transversality condition is not imposed, the model can have any other solution
  \begin{align}
    y_t = y_t^* + b_t
  \end{align}
$b_t$ is bubble component; model must satisfy
\begin{align}
  y_t^* + b_t = x_t + a\mathbb{E}_t y^*_{t+1} + a\mathbb{E}_tb_{t+1} 
\end{align}
By construction we have that
\begin{align}
  y_t^* = x_t + a\mathbb{E}_t y^*_{t+1}
\end{align}
This entails that the additional components will satisfy
\begin{align}
  b_t = a\mathbb{E}_tb_{t+1} 
\end{align}
\end{frame}
%--------------------------------------

%--------------------------------------
\begin{frame}
  When $|a|<1$, $b$ is always expected to get bigger in absolute value, going to infinity in expectation: This is a bubble.
  \begin{align}
    b_t=\mathbb{E}_t \left[\frac{1}{1+r} \right] b_{t+1}
  \end{align}
  $b_t$ grows in expectations at a rate of $r$: There may be restrictions in the real economy that stop $b$ growing forever.


\end{frame}
%--------------------------------------

%--------------------------------------
\begin{frame}
  Can also satisfy 
  \begin{align}
    b_t = a\mathbb{E}_tb_{t+1} 
  \end{align}
  through
\begin{align}
  b_{t+1} \left\{\begin{matrix}
  (aq)^{-1}b_t + e_{t+1} & \;with\;probability\; &q
\\ e_{t+1} & \;with\;probability\;&1-q
\end{matrix}\right.   
\end{align}
\begin{align}
  \mathbb{E}_te_{t+1}=0
\end{align}
This is a bubble that everyone knows is going to crash eventually. 
Even in that case, the bubble can keep going. 
Imposing 
\begin{align}
  \lim_{N \rightarrow \infty} a^N\mathbb{E}_ty_{t+N}=0
\end{align}
rules out bubbles of this or any other form.
\end{frame}
%--------------------------------------

%--------------------------------------
\begin{frame}
  \begin{align}  y_t &= \sum^\infty_{k=0}a^k\mathbb{E}_tx_{t+k}  \end{align}
Provides some useful insights into how $y_t$ is determined. 
However, without some assumptions about how $x_t$ evolves over time, it cannot be used to give precise predictions about the dynamics of $y_t$. 
Ideally we want to simulate the behaviour of $y_t$. 
One reason there is a strong link between DSGE modeling and VAR models is that the question on the behaviour of $y_t$ is often  addressed by assuming that the exogenous variables $x_t$ are generated by backward-looking time series models, like VARs.
\end{frame}
%--------------------------------------

%--------------------------------------
\begin{frame}
  \textbf{Structural to reduced form}
  \begin{align}
    y_t &= \sum^\infty_{k=0}a^k\mathbb{E}_tx_{t+k}
  \end{align}
  Provide insights into determination of $y_t$ but
  \begin{itemize}
    \item Need assumption about how $x_t$ evolves over time
    \item Required for predictions about $y_t$ dynamics
  \end{itemize}
  Often assumed that $x_t$ is generated by backward looking process: Can be modeled with VAR 
\end{frame}
%--------------------------------------

%--------------------------------------
\begin{frame}
 Consider $x_t$ with DGP  
\begin{align}
  x_t&=\rho x_{t-1} + \epsilon_t \\ \nonumber 
  |\rho| &< 1
\end{align}
\begin{align}
  \mathbb{E}_tx_{t+k}=\rho^kx_t
\end{align}
 Model solution can be written as 
\begin{align}
  y_t=\left[ \sum^{\infty}_{k=0}(a\rho)^k \right]x_t
\end{align}
\end{frame}
%--------------------------------------

%--------------------------------------
\begin{frame} 
  Assuming $|a\rho|<1$, the infinite sum converges to
  \begin{align}
  \sum^{\infty}_{k=0}(a\rho)^k=\frac{1}{1-a\rho}
\end{align}

The model solution becomes
\begin{align}
  y_t=\frac{1}{1-a\rho}x_t
\end{align}
This is the reduced-form solution for the model which, together with the equation describing the evolution of $x_t$, can be easily simulated on a computer. 
\end{frame}
%--------------------------------------

%--------------------------------------
\begin{frame}
  \textbf{Reduced-form} model has VAR-like representation
  \begin{align}
  y_t &= \frac{1}{1-a\rho}(\rho x_{t-1}+\epsilon_t)\\
      &= \rho y_{t-1} +  \frac{1}{1-a\rho}\epsilon_t
  \end{align}
  Both the $x_t$ and $y_t$ series have purely backward-looking representations
  \begin{itemize}
    \item This simple model helps to explain how theoretical models tend to predict that the data can be described well using a VAR.
  \end{itemize}
\end{frame}
%--------------------------------------

%--------------------------------------
\begin{frame}
  \textbf{DSGE recipe}
  \begin{enumerate}
  \item Obtain structural equations involving expectations of future driving variables (the $\mathbb{E}_tx_{t+k}$ terms)
  \item Make assumptions about the time series process for the driving variables ($x_t$)
  \item Solve for a reduced-form solution that can be simulated on a computer
\end{enumerate}
\end{frame}
%--------------------------------------

%--------------------------------------
\begin{frame}
  \textbf{Permanent income hypothesis:} Let consumption depend on present discounted value of after-tax income
\begin{align}
  c_t=\gamma \sum^{\infty}_{k=0}\beta^k \mathbb{E}_ty_{t+k}
\end{align}
 Income generating process is 
 \begin{align}
   y_t&= (1+g)Y_{t-1}+\epsilon_t \\
   \mathbb{E}_ty_{t+k}&=(1+g)^ky_t
 \end{align}
 Reduced-form representation
 \begin{align}
    c_t=\gamma \left[\sum^{\infty}_{k=0} (\beta(1+g))^k \right ]y_t
  \end{align}
\end{frame}
%--------------------------------------

%--------------------------------------
\begin{frame}
  Assuming 
  \begin{align}
    \beta(1+g)^k<1
  \end{align}
  this becomes
\begin{align}
  c_t=\frac{\gamma}{1-\beta(1+g)}y_t
\end{align}
Two important implications
  \begin{enumerate}
  \item The structural equation is always true for the model
  \item The reduced-representation depends on the process $y_t$ taking a particular form
\end{enumerate}
If the process changes, so will the reduced-form process. 
\end{frame}
%--------------------------------------

%--------------------------------------
\begin{frame}
  \textbf{Temporary tax cuts}
  Suppose that the government is thinking about a temporary one-period income tax cut. 
They ask their advisers what the estimated effect of the tax cut on consumption would be. 
So they run a regression of consumption $c_t$ on after-tax income $y_t$. 
\begin{itemize}
  \item If, in the past consumers had generally expected income growth $g$, then the income coefficient will be approximately $\frac{\gamma}{1-\beta(1+g)}$: each 1 increase in income produced by the tax cut will increase consumption by $\frac{\gamma}{1-\beta(1+g)}$
  \item If the households have rational expectations the increase will be just $\gamma$
\end{itemize}
In the case were $\beta=0.95$ and $g=0.02$ the temporary tax cut would lead to an increase of $32\gamma$ in spending increase of $32\gamma$ whereas under rational expectations it would just be $\gamma$.
\end{frame}
%--------------------------------------

%--------------------------------------
\begin{frame}
  \textbf{Jump variables}
  \begin{align}  
      y_t=\sum^{\infty}_{k=0}a^k\mathbb{E}_tx_{t+k} 
  \end{align}
  These are known as \textbf{jump variables}
  \begin{itemize}
  \item Only depend on what happens today and what's expected to happen tomorrow
  \item Will jump if expectations about future change
  \item Past does not restrict their movement
\end{itemize}
\end{frame}
%--------------------------------------

%--------------------------------------
\begin{frame}
  \textbf{Second-order stochastic difference equations}
  Jump variables are ok for stock prices but less for variables in real economy; these take on form of second-order stochastic difference equations
  \begin{align}  
    y_t=ay_{t-1} + b\mathbb{E}_ty_{t+1} + x_t 
  \end{align}
  How to solve these? Suppose there is value for $\lambda$ such that
  \begin{align}
    v_t=y_t-\lambda y_{t-1}  
  \end{align}
  follows SDE of form
  \begin{align}
  v_t=\alpha \mathbb{E}_tv_{t+1} + \beta x_t  
\end{align}
\end{frame}
%--------------------------------------



%--------------------------------------
\begin{frame}
  \begin{align}
  v_t=\alpha \mathbb{E}_tv_{t+1} + \beta x_t  
  \end{align}
  Can solve for $v_t$ and back out the values for $y_t$. 
  Given 
  \begin{align}
    y_t=v_t+\lambda y_{t-1}
  \end{align}
  Rewrite (X) as
\begin{align}
  v_t+\lambda y_{t-1} &= ay_{t-1} + b(\mathbb{E}_tv_{t+1} + \lambda y_t) + x_t\\ \nonumber
  &= ay_{t-1} + b\mathbb{E}_tv_{t+1} + b\lambda(v_t + \lambda y_{t-1}) + x_t\\ \nonumber
  (1-b\lambda)v_t &= b\mathbb{E}_tv_{t+1} + x_t + (b\lambda^2 - \lambda +a)y_{t-1}
\end{align}
\end{frame}
%--------------------------------------

%--------------------------------------
\begin{frame}
  By definition, $\lambda$ is a number such that the $v_t$  it defined followed a first-order stochastic difference equation; $\lambda$ satisfies 
  \begin{align}
    b\lambda^2 - \lambda + a=0
  \end{align}
Two values for $\lambda$ that satisfy it; for each value we can characterize $v_t$ by
\begin{align}
  v_t &= \frac{b}{1-b\lambda}\mathbb{E}_t v_{t+1} + \frac{1}{1-b\lambda}x_t\\
  &= \frac{1}{1-b\lambda}\sum^{\infty}_{k=0}\left(\frac{b}{1-b\lambda}\right)^k \mathbb{E}_t x_{t+k}
\end{align}
$y_t$ obeys
\begin{align}
  y_t=\lambda y_{t-1} + \frac{1}{1-b\lambda}\sum^{\infty}_{k=0}\left(\frac{b}{1-b\lambda}\right)^k E_t x_{t+k} 
\end{align}
Usually only one of the potential values of $\lambda$ is less than one in absolute value, so this gives the unique stable solution.
\end{frame}
%--------------------------------------

%--------------------------------------
\begin{frame}
  Solution for single equation can be generalised to vectors
  \begin{align}
    Z_t&=\begin{pmatrix}      z_{1t} \\ z_{2t} \\ . \\ z_{nt}  \end{pmatrix}     
\end{align}
 Can put this in macroeconomic model
 \begin{align}
   Z_t&=B\mathbb{E}_tZ_{t+1} + X_t
 \end{align}
$B$ is an $n\; x\; n$ matrix. Can use repeated substitution to provide solution
  \begin{align}  
  Z_t=\sum^{\infty}_{k=0}B^k \mathbb{E}_t X_{t+k}  
\end{align}
\end{frame}
%--------------------------------------

%--------------------------------------
\begin{frame}
  \textbf{Eigenvalues:} Eq. 28 will only give stable solution under certain conditions
  $\lambda_i$ is an eigenvalue of the matrix $B$ if there exists a vector $e_i$ such that 
  \begin{align}
     Be_i=\lambda_i e_i 
  \end{align}
  Denote by $P$ the matrix that has as columns $n$ eigenvectors corresponding to these eigenvalues. 
\begin{align}
  \Omega=\begin{pmatrix}
    \lambda_1 &0 &0 &0\\
    0 &\lambda_2 &0 &0\\
    0 & 0 & . &0\\
    0 &0 &0 & \lambda_n
  \end{pmatrix}
\end{align}
$\Omega$ is a diagonal matrix of eigenvalues. 
\end{frame}
%--------------------------------------

%--------------------------------------
\begin{frame}
  \begin{align}
    BP&=P\Omega \\ \nonumber
    B&=P\Omega P^{-1}
  \end{align}
  Provides information on relationship between eigenvalues and higher powers of $B$
\begin{align}
  B^n=P\Omega P^{-1} = P \begin{pmatrix}
    \lambda_1 &0 &0 &0\\
    0 &\lambda_2 &0 &0\\
    0 & 0 & . &0\\
    0 &0 &0 & \lambda_n
  \end{pmatrix}
  P^{-1}
\end{align}
\end{frame}
%--------------------------------------

%--------------------------------------
\begin{frame}
\begin{itemize}
  \item The difference between lower and higher powers of $B$ is that the higher powers depend on the eigenvalues take to the power $n$
  \item If all of the eigenvalues are in the unit circle, i.e. less than one in absolute value, then all of the entries in $B^n$ will tend towards zero as $n \rightarrow \infty$
\end{itemize}
  \begin{align}
    Z_t=B\mathbb{E}_tZ_{t+1} + X_t
  \end{align}
  Has a unique stable forward-looking solution is that the eigenvalues of $B$ are all inside the unit circle.   
\end{frame}
%--------------------------------------

%--------------------------------------
\begin{frame}
  \textbf{Calculating eigenvalues}
  \begin{align}
    A = \begin{pmatrix}    a_{11} & a_{12} \\ a_{21} & a_{22}  \end{pmatrix}
  \end{align}
  Suppose that $A$ has two eigenvalues, $\lambda_1$ and $\lambda_2$
\begin{align}
  \lambda = \begin{pmatrix}
    \lambda_1 \\ \lambda_2
  \end{pmatrix}
\end{align}
The fact that there are eigenvectors when multiplied by $A-\lambda I$ equal a vector of zeroes means that the determinant of the matrix equal zero.
\begin{align}
  A-\lambda I = \begin{pmatrix}
    a_{11}-\lambda & a_{12} \\ a_{21} & a_{22}-\lambda_2
  \end{pmatrix}
\end{align}
Solving the quadratic formula will give the two eigenvalues of A
\begin{align}
  (a_{11}-\lambda_1 a_{12})(a_{22}-\lambda_2)-a_{12}a_{21}=0
\end{align}
\end{frame}
%--------------------------------------


%--------------------------------------
\begin{frame}
  \textbf{Binder-Pesaran}  
  \begin{align}  
  Z_t=AZ_{t-1} + B\mathbb{E}_t Z_{t+1} + HX_t 
  \end{align}
  Binder and Pesaran solved this model in a manner exactly analogous to the second-order SDE 
  \begin{align}
   W_t=Z_t - CZ_{t-1}  
   \end{align}
   Find matrix $C$ such that (X) obeys a first-order matrix of the form 
   \begin{align}
    W_t=F\mathbb{E}_t W_{t+1} + GX_t  
    \end{align}
    Transform the problem of solving the second-order system in equation into a simpler first-order system.
\end{frame}
%--------------------------------------

%--------------------------------------
\begin{frame} 
 What must $C$ be?
 \begin{align}
   Z_t=W_t + CZ_{t-1}
 \end{align}
 Can re-write as 
 \begin{align}
  W_t + CZ_{t-1} &= AZ_{t-1} + B(\mathbb{E}_t W_{t+1} + CZ_t) + HX_t\\ \nonumber
  &= AZ_{t-1} + B(\mathbb{E}_t W_{t+1} + C(W_t+CZ_{t-1})) + HX_t  
  \end{align}
  $C$ will eliminate the backward-looking terms; can re-arrange
  \begin{align}
    (I-BC)W_t = B\mathbb{E}_t W_{t+1} + (BC^2-C+A)Z_{t-1} + HX_t  
  \end{align}
\end{frame}
%--------------------------------------

%--------------------------------------
\begin{frame}
  $C$ makes $W_t$ follow a first-order forward-looking matrix equation, i.e. no extra $Z_{t-1}$ terms, we get
\begin{align}
  BC^2-C+A=0
\end{align}
This equation can be solved to give $C$, one method is to use
\begin{align}
  C=BC^2+A
\end{align}
\begin{enumerate}
  \item Provide initial guess $C_0=I$
  \item Iterate on $C_n=BC^2_{n-1}$ until all entries in $C_n$ converge
\end{enumerate}  
\end{frame}
%--------------------------------------

%--------------------------------------
\begin{frame}
  Knowing $C$ we have
  \begin{align}
    W_t=F\mathbb{E}_t W_{t+1} + GX_t
  \end{align}
  where
  \begin{align}
      F &= (I-BC)^{-1}B\\
      G &= (I-BC)^{-1}H  
  \end{align}
  Assuming that all eigenvalues of $F$ are inside the unit circle, this has a stable forward-looking solution
  \begin{align}
    W_t= \sum^{\infty}_{k=0} F^k\mathbb{E}_t(GX_{t+k})
  \end{align}
  Which can be written in terms of the original equation as
  \begin{align}
    Z_t= CZ_{t-1} + \sum^{\infty}_{k=0} F^k\mathbb{E}_t(GX_{t+k})
  \end{align}
\end{frame}
%--------------------------------------


%--------------------------------------
\begin{frame}
  \textbf{Reduced form representation:}
  Suppose variables $X_t$ can be represented using VAR
  \begin{align}
    X_t=DX_{t-1} + \epsilon_t
  \end{align}
  Let eigenvalues for $D$ be inside unit circle, this implies
  \begin{align}
    \mathbb{E}_t X_{t+k}=D^k X_t
  \end{align}
  Model solution is given by
  \begin{align}
    Z_t= CZ_{t-1} + \left[ \sum^{\infty}_{k=0}F^k GD^k \right]X_t  
  \end{align}
\end{frame}
%--------------------------------------

%--------------------------------------
\begin{frame}
  The infinite sum in eq. (X)to matrix $P$; model has a reduced form representation which can be simulated along with the VAR process 
\begin{align}
  Z_t= CZ_{t-1} + PX_t
\end{align}
This provides a relatively simple recipe for simulating DSGE models:
\begin{enumerate}
  \item Specify matrices $A,B,H$
  \item Solve for $C,F,G$
  \item Specify a VAR process for the driving variables
  \item Obtain the reduced-form representations  
\end{enumerate}
\end{frame}
%--------------------------------------

%--------------------------------------
\begin{frame}
  There are often multiple values for different variables at $t$: estimate model as
  \begin{align}
    KZ_t=AZ_{t-1} + B\mathbb{E}_t Z_{t+1} + HX_t
  \end{align}
  Computer software will multiply both sides by $K^{-1}$  and solve using the Binder-Pesaran model. 
\begin{align}
  Z_t= K^{-1}AZ_{t-1} + K^{-1}B\mathbb{E}_t Z_{t+1} + K^{-1}HX_t
\end{align}
 Only need to figure out what model implies in terms of $K,A,B,H$; computer will provide following representation
  \begin{align}
    Z_t &= CZ_{t-1} + PX_t\\
    X_t &=DX_{t-1} + \epsilon_t
  \end{align}
\end{frame}
%--------------------------------------


%--------------------------------------
\end{document}

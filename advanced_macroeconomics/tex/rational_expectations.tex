\documentclass{beamer}
\usetheme{}
\usecolortheme{dolphin}           
\useinnertheme{circles}
\setbeamertemplate{itemize items}[default]
\setbeamertemplate{enumerate items}[default]
\usepackage[T1]{fontenc}
\usepackage[utf8]{inputenc}
\usepackage{lmodern}
\usepackage{amsmath}
\usepackage{booktabs} 
\usepackage{graphicx}        
\usepackage{array}
\usepackage{color}
\makeatletter
\def\zapcolorreset{\let\reset@color\relax\ignorespaces}
\def\colorrows#1{\noalign{\aftergroup\zapcolorreset#1}\ignorespaces}
\makeatother
\setbeamertemplate{navigation symbols}{}
\setbeamertemplate{footline}[frame number]

%--------------------------------------
\title{Rational expectations}
\author{School of Economics, University College Dublin}
\date{Spring 2018}
\begin{document}

%--------------------------------------
\begin{frame}
 \titlepage
\end{frame}
%--------------------------------------

%--------------------------------------
\begin{frame}
  \textbf{Lucas} (1976)
  \begin{quote}
  Given that the structure of an econometric model consists of optimal decision rules of economic agents, and that optimal decision rules vary systematically with changes in the structure of series relevant to the decision maker, it follows that any change in policy will systematically alter the structure of econometric models.
  \end{quote}
\end{frame}
%--------------------------------------

%--------------------------------------
\begin{frame}
  \begin{align}
    y_t &= a \tilde{y}_{t+1} + bx_t\\ \nonumber
    |a|&<1;b\neq0
  \end{align}
  $\tilde{y}_{t+1}$ is expectation of $y_{t+1}$: how to specify?
  \begin{align}
    \tilde{y}_{t+1}=y_t
  \end{align}
  \begin{align}
    y_t &= a y_t + bx_t\\ \nonumber
    y_t&= \frac{b}{1-a}x_t
  \end{align}
\end{frame}
%--------------------------------------

%--------------------------------------
\begin{frame}
  \begin{align}
    \tilde{y}_{t+1} = \mathbb{E}_ty_{t+1}
  \end{align}
  \begin{align}
  y_t &= a \tilde{y}_{t+1} + bx_t\\ \nonumber
  y_t &= a \mathbb{E}_ty_{t+1} + bx_t\\     
  \end{align}
\end{frame}
%--------------------------------------

%--------------------------------------
\begin{frame}
  \textbf{DSGE} models have both backward and forward looking elements:
  Backward looking dynamics come from identities
  \begin{align}
     K_t= (1-\gamma) K_{t-1} + I_t
   \end{align} 
  Forward looking dynamics from optimising behaviour
  \begin{itemize}
    \item What agents expect to happen tomorrow is very important for what they decide to do today.
  \end{itemize}
  Modeling this requires assumptions about how people formulate expectations
  \begin{itemize}
    \item DSGE approach relies on the idea that people have \textbf{rational expectations}
  \end{itemize}
\end{frame}
%--------------------------------------

%--------------------------------------
\begin{frame}
  \textbf{Rational expectations} 
   \begin{quote}
     Agents formulate expectations in such a way that their subjective probability distribution of economic variables - conditional on the available information - coincides with the objective probability distribution of the same variable - according to a measure of the state of nature- in an equilibrium.
   \end{quote}
   \medskip
   In other words agents
    \begin{enumerate}
    \item Use publicly available information efficiently (i.e. no systematic mistakes)
    \item Understand the structure of the economy and base their expectations on this knowledge
  \end{enumerate}
\end{frame}
%--------------------------------------

%--------------------------------------
\begin{frame}
  Rational expectations (RE) is a \textbf{baseline} assumption about people's behaviour.
  \begin{align}
    y_t=x_t + a\mathbb{E}_ty_{t+1}
  \end{align}
  Today's value of $y$ is determined by  
\begin{enumerate}
  \item $x$
  \item Tomorrow's expected value of $y$
\end{enumerate}
\end{frame}
%--------------------------------------

%--------------------------------------
\begin{frame}
  Under RE agents formulate expectations that is consistent with  
\begin{align}
  \mathbb{E}_ty_{t+1} &= \mathbb{E}_tx_{t+1} + a\mathbb{E}_t\mathbb{E}_{t+1}y_{t+2}\\ \nonumber
  \mathbb{E}_ty_{t+1} &= \mathbb{E}_tx_{t+1} + a\mathbb{E}_ty_{t+2}
\end{align}
\begin{align}
  \mathbb{E}_t\mathbb{E}_{t+1}y_{t+2} = \mathbb{E}_ty_{t+2}
\end{align}
\medskip
This is the \textbf{Law of Iterated Expectations}
\begin{itemize}
  \item It is not rational for me to expect to have a different expectation next period for $y_{t+2}$ than the one that I have today (based on the information set available at $t$).  
\end{itemize}
\end{frame}
%--------------------------------------

%--------------------------------------
\begin{frame}
  Use repeated substitution to get solution for model  
\begin{align}  
    y_t = x_t + a\mathbb{E}_tx_{t+1} + a^2\mathbb{E}_ty_{t+2} 
\end{align}
Include additional periods
\begin{align}
  y_t &= x_t + a\mathbb{E}_tx_{t+1} + ...+ a^{N-1}\mathbb{E}_tx_{t+N-1} +a^{N}\mathbb{E}_ty_{t+N}\\ \nonumber
      &= \sum^{N-1}_{k=0}a^k\mathbb{E}_tx_{t+k} + a^{N}\mathbb{E}_ty_{t+N}\\ \nonumber      
\end{align}
 Assuming
 \begin{align}
   \lim_{N \rightarrow \infty} a^{N}\mathbb{E}_ty_{t+N} = 0
 \end{align}
 We get
 \begin{align}
   y_t=\sum^\infty_{k=0}a^k\mathbb{E}_tx_{t+k}
 \end{align}
\end{frame}
%--------------------------------------

%--------------------------------------
\begin{frame}
  \textbf{Asset pricing}\\
  \begin{enumerate}
    \item Asset with price $p_t$, yielding dividend $d_t$
    \item Close alternative with guaranteed rate of return $r$
  \end{enumerate}
  \medskip  
  Risk neutral investor will only hold the asset if the yield is the same as the rate of return
\begin{align}  
  \frac{d_t + \mathbb{E}_tp_{t+1}}{p_t}&=1+r  
\end{align}
\end{frame}
%--------------------------------------

%--------------------------------------
\begin{frame}
 Rearranging 
 \begin{align}  
  \frac{d_t + \mathbb{E}_tp_{t+1}}{p_t}&=1+r  
\end{align}
Gives
\begin{align}
  p_t&= \frac{d_t}{1+r} + \frac{\mathbb{E}_tp_{t+1}}{1+r}
\end{align}
Repeated substitution solution
\begin{align}
 p_t&= \sum^{\infty}_{k=0}\left(\frac{1}{1+r}\right)^{k+1}\mathbb{E}_td_{t+k}     
\end{align}
\medskip
 \textbf{Dividend-discount model}: Asset prices should equal a discounted present-value sum of expected future dividends.
\end{frame}
%--------------------------------------

%--------------------------------------
\begin{frame}
  \textbf{Backward solution}  
 \begin{align}
    y_t=x_t + a\mathbb{E}_ty_{t+1} \\ \nonumber
    y_t= x_t + ay_{t+1} + a\epsilon_{t+1}
 \end{align}
  Forecast error $\epsilon_{t+1}$ cannot be predicted at time $t$.\\
  Can move time index back one period
  \begin{align}
  y_{t-1} &= x_{t-1} + ay_t + a\epsilon_t\\ \nonumber
  ay_t &= y_{t-1} - x_{t-1} - a\epsilon_t\\ \nonumber
  y_t &= a^{-1}y_{t-1} - a^{-1}x_{t-1} - \epsilon_t 
\end{align}
\begin{align}
  y_t&= -\sum^{\infty}_{k=0}a^{-k}\epsilon_{t-k}-\sum^{\infty}_{k=1}a^{-k}x_{t-k}
\end{align}
\end{frame}
%--------------------------------------

%--------------------------------------
\begin{frame}
  \textbf{Forward \& backward}
  \begin{align*}
    y_t &= \sum^\infty_{k=0}a^k\mathbb{E}_tx_{t+k} \\
    y_t&= -\sum^{\infty}_{k=0}a^{-k}\epsilon_{t-k}-\sum^{\infty}_{k=1}a^{-k}x_{t-k}
  \end{align*}
  \medskip
  Both are correct; which one we use depends on $a$  
\end{frame}
%--------------------------------------

%--------------------------------------
\begin{frame}
\begin{align*}
    |a|>1
  \end{align*}
  Weights on forward solution are explosive
  \begin{itemize}
    \item Forward solution will not converge to finite sum
    \item $y_t$ depends more on values of $x_t$ far in the distant future than on today's values
  \end{itemize}  
  $\Rightarrow$ Use backward solution; will be indeterminate though
  \begin{align}
    \mathbb{E}_{t-1}\epsilon_t=0
  \end{align}
  \medskip
  Cannot predict $y_t$ even if we know full path of $x_t$
\end{frame}
%--------------------------------------

%--------------------------------------
\begin{frame} 
  \begin{align*}
    |a|<1
  \end{align*}
  \medskip
  Weights on backward solution are explosive; use forward solution
  \begin{itemize}
    \item Path of $x_t$ will tell path of $y_t$
  \end{itemize}
  \medskip
  Most cases assume that $|a|<1$
  \begin{align}
    \lim_{N \rightarrow \infty} a^N \mathbb{E}_t y_{t+N} = 0
  \end{align}
  \medskip
   $y_t$ can't grow too fast (\textit{transversality condition}) 
\end{frame}
%--------------------------------------

%--------------------------------------
\begin{frame}
  \textbf{Rational bubbles}\\
   If transversality condition is not imposed, model can have any other solution
  \begin{align}
    y_t^* = \sum^{\infty}_{k=0}a^k\mathbb{E}_tx_{t+k}
  \end{align}
   Consider other solution
  \begin{align}
    y_t = y_t^* + b_t
  \end{align}
  \medskip
  With bubble component $b_t$ model must satisfy
\begin{align}
  y_t^* + b_t = x_t + a\mathbb{E}_t y^*_{t+1} + a\mathbb{E}_tb_{t+1} 
\end{align}
\end{frame}
%--------------------------------------

%--------------------------------------
\begin{frame}
Model is by construction
\begin{align}
  y_t^* = x_t + a\mathbb{E}_t y^*_{t+1}
\end{align}
\medskip
Bubble component will satisfy
\begin{align}
  b_t = a\mathbb{E}_tb_{t+1} 
\end{align}
\medskip
  Because $|a|<1$, $|b|$ is always expected to get larger
  \begin{align}
    b_t=\mathbb{E}_t \left[\frac{1}{1+r} b_{t+1}\right]
  \end{align}
  \medskip
  $b_t$ grows in expectations at rate $r$   
\end{frame}
%--------------------------------------

%--------------------------------------
\begin{frame}
There may be restrictions in the real economy that stop $b$ growing forever
 \begin{itemize}
   \item e.g. constant growth
 \end{itemize}
\medskip
Following would do the trick
\begin{align}
  b_{t+1} \left\{\begin{matrix}
  (aq)^{-1}b_t + e_{t+1} &with\; &\Pr(q)
\\ e_{t+1} & \;with\;&Pr(1-q)
\end{matrix}\right.   
\end{align}
With 
\begin{align}
  \mathbb{E}_te_{t+1}=0
\end{align}
\medskip
Imposing 
\begin{align}
  \lim_{N \rightarrow \infty} a^N\mathbb{E}_ty_{t+N}=0
\end{align}
rules out bubbles of this or any other form.
\end{frame}
%--------------------------------------

%--------------------------------------
\begin{frame}
\textbf{Structural to reduced form}
  \begin{align}  y_t &= \sum^\infty_{k=0}a^k\mathbb{E}_tx_{t+k}  \end{align}
 Requires assumption on how $x_t$ evolves over time
\begin{itemize}
   \item Otherwise cannot be used for predictions on $y_t$ dynamics
\end{itemize}
\medskip
Assumption that exogenous variables $x_t$ are generated by backward-looking time-series models
\begin{itemize}
  \item Therefore strong link between DSGE and VAR models
\end{itemize}
\end{frame}
%--------------------------------------

%--------------------------------------
\begin{frame}
 Consider $x_t$ with DGP  
\begin{align}
  x_t&=\rho x_{t-1} + \epsilon_t \\ \nonumber 
  |\rho| &< 1
\end{align}
\begin{align}
  \mathbb{E}_tx_{t+k}=\rho^kx_t
\end{align}
 Model solution given by
\begin{align}
  y_t=\left[ \sum^{\infty}_{k=0}(a\rho)^k \right]x_t
\end{align}
\end{frame}
%--------------------------------------

%--------------------------------------
\begin{frame} 
  Assuming $|a\rho|<1$: infinite sum converges to
  \begin{align}
  \sum^{\infty}_{k=0}(a\rho)^k=\frac{1}{1-a\rho}
\end{align}

Gives reduced-form solution
\begin{align}
  y_t=\frac{1}{1-a\rho}x_t
\end{align}
\medskip
Can be combined with equation describing $x_t$ and simulated on computer
\end{frame}
%--------------------------------------

%--------------------------------------
\begin{frame}
  \textbf{Reduced-form} model has VAR-like representation
  \begin{align}
  y_t &= \frac{1}{1-a\rho}(\rho x_{t-1}+\epsilon_t)\\
      &= \rho y_{t-1} +  \frac{1}{1-a\rho}\epsilon_t
  \end{align}
  \medskip
  $x_t,y_t$ series have purely backward-looking representations
  \begin{itemize}
    \item Theoretical models tend to predict that data can be described by VAR
  \end{itemize}
\end{frame}
%--------------------------------------

%--------------------------------------
\begin{frame}
  \textbf{DSGE recipe}
  \begin{enumerate}
  \item Obtain structural equations involving expectations of future driving variables
  \begin{align*}
    \mathbb{E}_tx_{t+k}
  \end{align*}
  \item Make assumptions about time-series process for the driving variables 
  \begin{align*}
    x_t
  \end{align*}
  \item Solve for reduced-form solution that can be simulated on a computer
  \begin{align*}
    y_t=\frac{1}{1-a\rho}x_t
  \end{align*}
\end{enumerate}
\end{frame}
%--------------------------------------

%--------------------------------------
\begin{frame}
  \textbf{Permanent income hypothesis}\\
  Let consumption depend on present discounted value of after-tax income
\begin{align}
  c_t=\gamma \sum^{\infty}_{k=0}\beta^k \mathbb{E}_ty_{t+k}
\end{align}
\medskip
 Income generating process is 
 \begin{align}
   y_t&= (1+g)y_{t-1}+\epsilon_t \\
   \mathbb{E}_ty_{t+k}&=(1+g)^ky_t
 \end{align}
 \medskip
 Reduced-form representation
 \begin{align}
    c_t=\gamma \left[\sum^{\infty}_{k=0} (\beta(1+g))^k \right ]y_t
  \end{align}
  \medskip
  Assuming $\beta(1+g)^k<1$ this becomes  
\begin{align}
  c_t=\frac{\gamma}{1-\beta(1+g)}y_t
\end{align}
\end{frame}
%--------------------------------------

%--------------------------------------
\begin{frame}  
\textbf{Lucas critique}\\
Two important implications
  \begin{enumerate}
  \item Structural equation is always true for the model
  \begin{align*}
    c_t=\gamma \sum^{\infty}_{k=0}\beta^k \mathbb{E}_ty_{t+k}
  \end{align*}
  \item Reduced-form depends on process $y_t$ taking particular form
  \begin{align*}
    c_t=\frac{\gamma}{1-\beta(1+g)}y_t
  \end{align*}
\end{enumerate}
\medskip
Reduced-form parameters change when $\mathbb{E}_ty_{t+k}$ changes
\begin{itemize}
    \item Reduced-form models using historical data useless for policy analysis
  \end{itemize}
\end{frame}
%--------------------------------------

%--------------------------------------
\begin{frame}
  \textbf{Temporary tax cut}\\
  Suppose following parameter values
  \begin{align*}
    g=0.02;\beta=0.95
  \end{align*}
  \medskip 
  If consumers expect $g$, tax cut will increase consumption by
  \begin{align*}
    \frac{\gamma}{1-\beta(1+g)}=\frac{\gamma}{1-0.95(1+0.02)}\approx 32\gamma
  \end{align*}
  \medskip
  Under rational expectations increase will be
  \begin{align*}
    \gamma
  \end{align*}
  \medskip
  Off by a factor of 32. 
\end{frame}
%--------------------------------------

%--------------------------------------
\begin{frame}
  \textbf{VAR vs. DSGE}\\
  VARs fit data well but cannot be used for policy analysis
  \begin{itemize}
    \item VARs do not allow reduced-form correlations to change over time
  \end{itemize}
  \medskip
  DSGE can as it includes policy equations based on rational expectations
  \begin{enumerate}[i]
      \item Relate interest rates to inflation and unemployment
      \item Dependence of fiscal variables on other macro variables
      \item Exchange rate regime
  \end{enumerate}    
  \medskip
  DSGE can explain pattern as result of structural changes in policy rules
\end{frame}
%--------------------------------------

%--------------------------------------
\begin{frame}
  \textbf{Jump variables}
  \begin{align*}  
      y_t=\sum^{\infty}_{k=0}a^k\mathbb{E}_tx_{t+k} 
  \end{align*}
  \begin{itemize}
  \item Depend on what happens today and what's expected to happen tomorrow
  \item Will jump if expectations about future change
  \item Past does not restrict their movement
\end{itemize}
\medskip
Useful to characterise stock prices
\end{frame}
%--------------------------------------

%--------------------------------------
\begin{frame}
  For real economy use \textbf{second-order stochastic difference equations}
  \begin{align}  
    y_t=ay_{t-1} + b\mathbb{E}_ty_{t+1} + x_t 
  \end{align}
  \medskip
  How to solve these? Suppose there is value for $\lambda$ such that
  \begin{align}
    v_t=y_t-\lambda y_{t-1}  
  \end{align}
  follows SDE of form
  \begin{align}
  v_t=\alpha \mathbb{E}_tv_{t+1} + \beta x_t  
\end{align}
\medskip
Can solve for $v_t$ and back out the values for $y_t$
\end{frame}
%--------------------------------------

%--------------------------------------
\begin{frame}
  Given 
  \begin{align}
    y_t=v_t+\lambda y_{t-1}
  \end{align}
  Rewrite as
\begin{align}
  v_t+\lambda y_{t-1} &= ay_{t-1} + b(\mathbb{E}_tv_{t+1} + \lambda y_t) + x_t\\ \nonumber
  &= ay_{t-1} + b\mathbb{E}_tv_{t+1} + b\lambda(v_t + \lambda y_{t-1}) + x_t\\ \nonumber  
\end{align}
 Rearrange to
 \begin{align}
   (1-b\lambda)v_t &= b\mathbb{E}_tv_{t+1} + x_t + (b\lambda^2 - \lambda +a)y_{t-1}
 \end{align}
\end{frame}
%--------------------------------------

%--------------------------------------
\begin{frame}
  By definition $\lambda$ is such that the $v_t$ it defined followed a first-order SDE: $\lambda$ satisfies 
  \begin{align}
    b\lambda^2 - \lambda + a=0
  \end{align}
  \medskip
  This is a quadratic equation so there are two values for $\lambda$ that satisfy it.
\end{frame}
%--------------------------------------

%--------------------------------------
\begin{frame}
 For either value can characterise $v_t$ by
 \begin{align}
  v_t &= \frac{b}{1-b\lambda}\mathbb{E}_t v_{t+1} + \frac{1}{1-b\lambda}x_t\\
  &= \frac{1}{1-b\lambda}\sum^{\infty}_{k=0}\left(\frac{b}{1-b\lambda}\right)^k \mathbb{E}_t x_{t+k}
\end{align}
$y_t$ obeys
\begin{align}
  y_t=\lambda y_{t-1} + \frac{1}{1-b\lambda}\sum^{\infty}_{k=0}\left(\frac{b}{1-b\lambda}\right)^k \mathbb{E}_t x_{t+k} 
\end{align}
 Usually have one potential value for which $|\lambda|<1$: gives unique stable solution
\end{frame}
%--------------------------------------

%--------------------------------------
\begin{frame}
  \textbf{Systems of RE equations:}\\
  \begin{enumerate}
    \item Generalise solution for single equation to vector
    \begin{align}
    Z_t&=\begin{pmatrix}      z_{1t} \\ z_{2t} \\ . \\ z_{nt}  \end{pmatrix}     
    \end{align}
    \item Specify as macroeconomic model
     \begin{align}
        Z_t&=B\mathbb{E}_tZ_{t+1} + X_t
     \end{align}   
     $B$ is an $n\; x\; n$ matrix
     \item Get solution using repeated substitution
     \begin{align}  
        Z_t=\sum^{\infty}_{k=0}B^k \mathbb{E}_t X_{t+k}  
      \end{align}
  \end{enumerate}  
  \textbf{NB-} Will give stable non-explosive solution under certain conditions
\end{frame}
%--------------------------------------

%--------------------------------------
\begin{frame}
  \textbf{Eigenvalues}\\
  $\lambda_i$ is an eigenvalue of the matrix $B$ if there exists a vector $e_i$ such that 
  \begin{align}
     Be_i=\lambda_i e_i 
  \end{align}
  \medskip
   Many $n\;x\;n$ matrices have $n$ eigenvalues
   \begin{itemize}
      \item Denote $P$ which has columns $n$ eigenvectors corresponding to these eigenvalues
    \end{itemize} 
   \begin{align}
     BP=P\Omega
   \end{align}
\begin{align}
  \Omega=\begin{pmatrix}
    \lambda_1 &0 &0 &0\\
    0 &\lambda_2 &0 &0\\
    0 & 0 & . &0\\
    0 &0 &0 & \lambda_n
  \end{pmatrix}
\end{align}
$\Omega$ is a diagonal matrix of eigenvalues. 
\end{frame}
%--------------------------------------

%--------------------------------------
\begin{frame}
  \textbf{Stability condition}
  \begin{align}
    BP&=P\Omega
  \end{align}
  Implies
  \begin{align}     
    B&=P\Omega P^{-1}
  \end{align}
 Provides information on relation between eigenvalues and higher powers of $B$
\begin{align}
  B^n=P\Omega^n P^{-1} = P \begin{pmatrix}
    \lambda_1^n &0 &0 &0\\
    0 &\lambda_2^n &0 &0\\
    0 & 0 & . &0\\
    0 &0 &0 & \lambda_n^n
  \end{pmatrix}
  P^{-1}
\end{align}
\medskip
Higher powers of $B$ depend on eigenvalues taken to power $n$
\end{frame}
%--------------------------------------

%--------------------------------------
\begin{frame}
  $B^n$ will tend towards zero as $n \rightarrow \infty$
  \begin{itemize}
    \item If all of the eigenvalues are in the unit circle $|\lambda|<1$
  \end{itemize}
  \medskip
  Therefore we can get unique stable forward-looking solution for 
  \begin{align}
    Z_t=B\mathbb{E}_tZ_{t+1} + X_t
  \end{align}
  if the eigenvalues of $B$ are all inside the unit circle.   
\end{frame}
%--------------------------------------

%--------------------------------------
\begin{frame}
  \textbf{Calculating eigenvalues}
  \begin{align}
    A = \begin{pmatrix}    a_{11} & a_{12} \\ a_{21} & a_{22}  \end{pmatrix}
  \end{align}
  Suppose two eigenvalues for $A$: $\lambda_1,\lambda_2$ 
\begin{align}
  \lambda = \begin{pmatrix}
    \lambda_1 \\ \lambda_2
  \end{pmatrix}
\end{align}
\medskip
There are eigenvectors when multiplied by $A-\lambda I$ equal a vector of zeroes: determinant of matrix 
\begin{align}
  A-\lambda I = \begin{pmatrix}
    a_{11}-\lambda & a_{12} \\ a_{21} & a_{22}-\lambda_2
  \end{pmatrix}
\end{align}
\medskip
will equal zero: solving quadratic formula will give the two eigenvalues of A
\begin{align}
  (a_{11}-\lambda_1 a_{12})(a_{22}-\lambda_2)-a_{12}a_{21}=0
\end{align}
\end{frame}
%--------------------------------------

%--------------------------------------
\begin{frame}
  \textbf{Binder-Pesaran method}  
  \begin{align}  
      Z_t=AZ_{t-1} + B\mathbb{E}_t Z_{t+1} + HX_t 
  \end{align}
  \medskip
  One-lag/one-lead form is only for illustration
  \begin{itemize}
    \item Equation summarises all possible linear RE models
  \end{itemize}
  \medskip
  Binder \& Pesaran (1996) solved this model in manner exactly analogous to second-order SDE, for
  \begin{align}
      W_t=Z_t - CZ_{t-1}  
   \end{align} 
   \medskip
   find matrix $C$ such that (70) obeys a first-order matrix 
   \begin{align}
      W_t=F\mathbb{E}_t W_{t+1} + GX_t  
  \end{align}
  \medskip
  i.e. turn second-order into simpler first-order 
\end{frame}
%--------------------------------------

%--------------------------------------
\begin{frame} 
  To determine $C$ can use fact that 
 \begin{align}
   Z_t=W_t + CZ_{t-1}
 \end{align}
 \medskip
 Can be rewritten as 
 \begin{align}
  W_t + CZ_{t-1} &= AZ_{t-1} + B(\mathbb{E}_t W_{t+1} + CZ_t) + HX_t\\ \nonumber
  &= AZ_{t-1} + B(\mathbb{E}_t W_{t+1} + C(W_t+CZ_{t-1})) + HX_t  
  \end{align}
  Rearranges to  
  \begin{align}
    (I-BC)W_t = B\mathbb{E}_t W_{t+1} + (BC^2-C+A)Z_{t-1} + HX_t  
  \end{align}
\end{frame}
%--------------------------------------

%--------------------------------------
\begin{frame}
  $C$ is matrix such that $W_t$ follow a first-order forward-looking matrix equation
  \begin{itemize}
    \item No extra $Z_{t-1}$ terms 
  \end{itemize}
  \medskip
  Follows that 
\begin{align}
  BC^2-C+A=0
\end{align}
\medskip
This equation can be solved to give $C$: which is non-trivial 
\end{frame}
%--------------------------------------

%--------------------------------------
\begin{frame}
  Can use 
  \begin{align}
  C=BC^2+A
\end{align}
and solve reiteratively:
\begin{enumerate}
  \item Provide initial guess $C_0=I$
  \item Iterate on $C_n=BC^2_{n-1}+A$ until all entries in $C_n$ converge
\end{enumerate}  
\end{frame}
%--------------------------------------


%--------------------------------------
\begin{frame}
  Once we know $C$, we have
  \begin{align}
    W_t=F\mathbb{E}_t W_{t+1} + GX_t
  \end{align}
  where
  \begin{align}
      F &= (I-BC)^{-1}B\\
      G &= (I-BC)^{-1}H  
  \end{align}
  \medskip
  Has a stable forward-looking solution
  \begin{itemize}
    \item Assuming all eigenvalues of $F$ are inside the unit circle
  \end{itemize}
  \begin{align}
    W_t= \sum^{\infty}_{k=0} F^k\mathbb{E}_t(GX_{t+k})
  \end{align}  
\end{frame}
%--------------------------------------


%--------------------------------------
\begin{frame}
  \textbf{Reduced form representation}\\
  Sub. (81) in original equation
  \begin{align}
    Z_t= CZ_{t-1} + \sum^{\infty}_{k=0} F^k\mathbb{E}_t(GX_{t+k})
  \end{align}
  \medskip
  Suppose variables $X_t$ can be represented using VAR
  \begin{align}
    X_t=DX_{t-1} + \epsilon_t
  \end{align}
  \medskip
  $D$ has eigenvalues inside unit circle, implies
  \begin{align}
    \mathbb{E}_t X_{t+k}=D^k X_t
  \end{align}  
\end{frame}
%--------------------------------------

%--------------------------------------
\begin{frame}
Model solution is given by
  \begin{align}
    Z_t= CZ_{t-1} + \left[ \sum^{\infty}_{k=0}F^k GD^k \right]X_t  
  \end{align}
  Infinite sum in (85) will converge to matrix $P$: model has reduced-form representation 
  \begin{itemize}
    \item Can be simulate along with the VAR process driving $X_t$
  \end{itemize}
\begin{align}
  Z_t= CZ_{t-1} + PX_t
\end{align} 
\end{frame}
%--------------------------------------

%--------------------------------------
\begin{frame}
  Recipe for simulating DSGE model:
\begin{enumerate}
  \item Specify matrices $A,B,H$
  \begin{align*}
    Z_t=AZ_{t-1} + B\mathbb{E}_t Z_{t+1} + HX_t 
  \end{align*}
  \item Solve for $C,F,G$
  \begin{align*}
    W_t&=Z_t - CZ_{t-1}\\
   W_t&=F\mathbb{E}_t W_{t+1} + GX_t  
  \end{align*}
  \item Specify a VAR process for the driving variables
  \begin{align*}
    X_t=DX_{t-1} + \epsilon_t
  \end{align*}
  \item Obtain the reduced-form representations  
  \begin{align*}
    Z_t= CZ_{t-1} + PX_t
  \end{align*}
\end{enumerate}
\end{frame}
%--------------------------------------






%--------------------------------------
\begin{frame} 
Finally, there are often multiple values for different variables at $t$: Can estimate
  \begin{align}
    KZ_t=AZ_{t-1} + B\mathbb{E}_t Z_{t+1} + HX_t
  \end{align}
  \medskip
  Computer will multiply both sides by $K^{-1}$  and solve using Binder-Pesaran model. 
\begin{align}
  Z_t= K^{-1}AZ_{t-1} + K^{-1}B\mathbb{E}_t Z_{t+1} + K^{-1}HX_t
\end{align}
\medskip
 Need to figure out what model implies for $K,A,B,H$; computer will provide following representation
  \begin{align}
    Z_t &= CZ_{t-1} + PX_t\\
    X_t &=DX_{t-1} + \epsilon_t
  \end{align}
  \medskip
  These can be used for further calculations
\end{frame}
%--------------------------------------


%--------------------------------------
\end{document}

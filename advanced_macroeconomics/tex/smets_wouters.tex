\documentclass{beamer}
\usetheme{}
\usecolortheme{dolphin}           
\useinnertheme{circles}
\setbeamertemplate{itemize items}[default]
\setbeamertemplate{enumerate items}[default]
\usepackage[T1]{fontenc}
\usepackage[utf8]{inputenc}
\usepackage{lmodern}
\usepackage{amsmath}
\usepackage{booktabs} 
\usepackage{graphicx}        
\usepackage{array}
\usepackage{color}
\makeatletter
\def\zapcolorreset{\let\reset@color\relax\ignorespaces}
\def\colorrows#1{\noalign{\aftergroup\zapcolorreset#1}\ignorespaces}
\makeatother
\graphicspath{{/home/swl/Dropbox/ucd/advanced_macro/figures/}} 
\setbeamertemplate{navigation symbols}{}
\setbeamertemplate{footline}[frame number]

%--------------------------------------
\title{Smets-Wouters model}
\author{School of Economics, University College Dublin}
\date{Spring 2018}
\begin{document}

%--------------------------------------
\begin{frame}
 \titlepage
\end{frame}
%--------------------------------------

%--------------------------------------
\begin{frame}
\textbf{Smets \& Wouters} (2007) "Shocks and Frictions in US Business Cycles : A Bayesian DSGE Approach"
 \scalebox{.7}{
  \begin{quote}
    Using a Bayesian likelihood approach, we estimate a dynamic stochastic general equilibrium model for the US economy using seven macroeconomic time series. The model incorporates many types of real and nominal frictions and seven types of structural shocks. We show that this model is able to compete with Bayesian Vector Autoregression models in out-of-sample prediction. We investigate the relative empirical importance of the various frictions. Finally, using the estimated model, we address a number of key issues in business cycle analysis: What are the sources of business cycle fluctuations? Can the model explain the cross correlation between output and inflation? What are the effects of productivity on hours worked? What are the sources of the “Great Moderation”?
  \end{quote}}
\end{frame}
%--------------------------------------


%--------------------------------------
\begin{frame}
 \textbf{Supply side}  
\begin{align}
  y_t=\phi_p(\alpha k_t^s + (1-\alpha)l_t + \epsilon_t^a)
\end{align}
$y_t$ is GDP\\
$k^s_t$ is capital in use: determined by lagged level of capital and a capacity utilisation variable
\begin{align}
  k_t^s = k_{t-1} + z_t
\end{align}
$l_t$ is labour input\\
$\epsilon_t^a$ is total factor productivity
\end{frame}
%--------------------------------------

%--------------------------------------
\begin{frame}
 $z_t$ linked to marginal productivity of capital; function of capital to labour ratio and the real wage
\begin{align}
  r_t^k = -(k_t-l_t) + w_t
\end{align}
\medskip
Total factor productivity evolves over time according to
\begin{align}
  \epsilon_t^a = \rho \epsilon_{t-1}^a + \eta_t^a
\end{align}
\end{frame}
%--------------------------------------

%--------------------------------------
\begin{frame}
 \textbf{Demand side}  
\begin{align}
  y_t = c_y c_t + i_y i_t + z_y z_t + \epsilon_t^g
\end{align}
$y_t$ is GDP\\
$c_t$ is consumption\\ 
$i_t$ is investment\\
$\epsilon_t^g$ is exogenous spending\\
\medskip
Variables with subscript $y$ are steady-state shares\\
$z_t$ is included in the resource constraint because of the assumption that there are costs associated with having high rates of capital utilisation. 
\end{frame}
%--------------------------------------

%--------------------------------------
\begin{frame}  
Exogenous spending is assumed to develop over time according to
\begin{align}
  \epsilon_t^g = \rho\epsilon_{t-1}^g + \eta_t^g + \rho_{ga}\eta_t^a
\end{align}
Exogenous spending is assumed to have two components
\begin{enumerate}
  \item Government spending
  \item Element related to productivity
\end{enumerate} 
\end{frame}
%--------------------------------------

%--------------------------------------
\begin{frame}
  \textbf{Consumption}  
\begin{align}
  c_t = c_1c_{t-1} + (1-c_1) E_t c_{t+1} + c_2(l_t-E_t l_{t+1}) - c_3(r_t - E_t \pi_{t+1} + \epsilon_t^b)
\end{align}
$c_1, c_2, c_3$ are constant parameters (functions of deeper structural parameters)\\
$r_t$ is the interest rate on a one-period safe bond (quarterly)\\ \medskip

$\epsilon_t^b$ is a risk premium shock determining the willingness of a household to hold the one-period bond
\begin{itemize}
  \item Preference shock that influence short-term consumption-saving decisions
\end{itemize}
\begin{align}
  e_t^b = \rho_b\epsilon_{t-1} + \eta_t^b
\end{align}
\end{frame}
%--------------------------------------

%--------------------------------------
\begin{frame}
  Two other important things concerning consumption equation
\begin{itemize}
  \item Backward looking consumption term represent habit forming
  \item Equation allows for substitution of consumption with labour input
\end{itemize}

\end{frame}
%--------------------------------------

%--------------------------------------
\begin{frame}
  \textbf{Investment}  
\begin{align}
  i_t = i_ti_{t-1} + (1-i_1)E_ti_{t+1} + i_2q_t + \epsilon_t^i
\end{align}
 Main driver of investment: $q_t$
\begin{align}
  q_t = q_1E_tq_{t+1} + (1-q_1)r_{t+1}^k - (r_t - E_t\pi_{t+1} + \epsilon_t^b)
\end{align}
\end{frame}
%--------------------------------------

%--------------------------------------
\begin{frame}
 \textbf{Prices}  
\begin{align}
  \mu_t^p = \alpha(k_t-l_t) + \epsilon_t^a - w_t
\end{align}
 Equation accounts for 
 \begin{enumerate}
   \item Diminishing marginal productivity of capital 
   \item Productivity shocks effect on costs
   \item Real wage
 \end{enumerate}
\end{frame}
%--------------------------------------

%--------------------------------------
\begin{frame}
  \textbf{Inflation}
\begin{align}
  \pi_t = \pi_1\pi_{t-1} +\pi_2 E_t\pi_{t+1} - \pi_3\mu_t^p + \epsilon_t^p
\end{align}
 New Keynesian Philips curve
 \begin{itemize}
   \item Adjusted to account for lagged inflation
   \item Most firms will index their prices based on past inflation levels and can only set an optimal price occasionally
 \end{itemize}
\medskip
$\epsilon_t^p$ is a price mark-up disturbance which is described by
\begin{align}
  \epsilon_t^p = \rho^p \epsilon^p_{t-1} + \eta_t^p - \mu_p\eta_{t-1}^p
\end{align}
\medskip
Shock affects both current and lagged inflation in order to get a temporary price level shock. 
\end{frame}
%--------------------------------------

%--------------------------------------
\begin{frame}
 \textbf{Wages}  
\begin{align}
  w_t = w_1w_{t-1} + (1-w_1)E_t(w_{t+1} + \pi_{t+1}) - \\ \nonumber
  w_2\pi_t + w_3\pi_{t-1} - w_t\mu_t^w + \epsilon_t^w   
\end{align}
\begin{align}
  \epsilon_t^w = \rho^w \epsilon^w_{t-1} + \eta_t^w - \mu_w \eta_{t-1}^w
\end{align}
\end{frame}
%--------------------------------------

%--------------------------------------
\begin{frame}
$\mu_t^w$ is the wage mark-up 
\begin{itemize}
  \item Gap between real wage and marginal rate of substitution between working and consuming
\end{itemize}
\begin{align}
  \mu_t^w &= w_t - mrs_t\\
  &= w_t - \left( \sigma l_t - \frac{1}{1-\lambda/\gamma} (c_t - \lambda c_{t-1}) \right)  
\end{align}
\\ \medskip
 Sort of sticky: wages adjust gradually to equate the marginal costs and benefits of working
\end{frame}
%--------------------------------------

%--------------------------------------
\begin{frame}
  \textbf{Monetary policy}
\begin{align}
  r_t = \rho r_{t-1} + (1-\rho)(r_\pi \pi_t + r_y(y_t - y_t^p)) + \\ \nonumber r_{\Delta y} [(y_t - y_t^p) - (y_{t-1} - y_{t-1}^p)] + \epsilon_t^r 
\end{align}
\begin{align}  \epsilon_t^r = \rho^r \epsilon^r_{t-1} + \eta_t^r \end{align}
\medskip
Central bank sets short-term interest rate according to
\begin{enumerate}
  \item Last period's interest rate
  \item Gradual adjustment towards target interest rate
  \begin{itemize}
    \item Depends on inflation and output gap
  \end{itemize}
  \item Output gap growth rate
\end{enumerate}
\medskip
Potential output defined as the level of output that would prevail if prices and wages were fully flexible
\begin{itemize}
  \item  Model effectively needs to be expanded to add shadow flexible-price economy
\end{itemize}
\end{frame}
%--------------------------------------

%--------------------------------------
\begin{frame}
 \textbf{VAR system}
\begin{align}
  Y_t = \begin{pmatrix}
    dlGDP_t \\ dlCONS_t \\ dlINV_t \\ dlWG_t \\ lHOURS_t \\ dlP_t \\ FEDFUNDS_t
  \end{pmatrix} =
  \begin{pmatrix}
    \overline{\gamma} \\ \overline{\gamma} \\ \overline{\gamma} \\ \overline{\gamma} \\ \overline{l} \\ \overline{\pi} \\ \overline{r}
  \end{pmatrix} +
  \begin{pmatrix}
    y_t-y_{t-1} \\c_t-c_{t-1} \\ i_t-i_{t-1} \\ w_t-w_{t-1} \\ l_t \\ \pi_t \\ r_t
  \end{pmatrix}  
\end{align}

\end{frame}
%--------------------------------------

%--------------------------------------
\begin{frame}
  Additional features compared to RBC or NK model:  
\begin{itemize}
  \item Adjustment costs for investment
  \item Capacity utilisation cost
  \item Habit persistence
  \item Price indexation
  \item Wage indexation
  \item All kinds of autocorrelated shock terms
\end{itemize}
\end{frame}
%--------------------------------------

%--------------------------------------
\begin{frame}
  Fixes included to overcome shortcomings previous model: slow things down
  \begin{itemize}
    \item Give random shocks longer lasting effects
    \item Make development of variables more sluggish
  \end{itemize}
  \medskip
  Velocity major shortcoming of RBC: wage/price indexation addresses NK shortcoming
  \begin{itemize}
    \item Failed to deal with inflation persistence
  \end{itemize}
  \medskip
  Adjustment are largely ad hoc: no clear theoretical grounding.
\end{frame}
%--------------------------------------

%--------------------------------------
\begin{frame}
  \begin{figure}
     \includegraphics[scale=.7]{sw_table1.eps}
   \end{figure} 
\end{frame}
%--------------------------------------

%--------------------------------------
\begin{frame}
  \begin{figure}
    \includegraphics[scale=.7]{sw_table1b.eps}
  \end{figure}
\end{frame}
%--------------------------------------

%--------------------------------------
\begin{frame}
  \begin{figure}
    \includegraphics[scale=.7]{sw_table3.eps}
  \end{figure}
\end{frame}
%--------------------------------------

%--------------------------------------
\begin{frame}
  \begin{figure}
    \includegraphics[scale=.8]{sw_figure1_gdp.eps}
  \end{figure}
\end{frame}
%--------------------------------------

%--------------------------------------
\begin{frame}
  \begin{figure}
    \includegraphics[scale=.8]{sw_figure2.eps}
  \end{figure}
\end{frame}
%--------------------------------------

%--------------------------------------
\begin{frame}
  \begin{figure}
    \includegraphics[scale=.8]{sw_figure3.eps}
  \end{figure}
\end{frame}
%--------------------------------------

%--------------------------------------
\begin{frame}
  \begin{figure}
    \includegraphics[scale=.8]{sw_figure4_gdp.eps}
  \end{figure}
\end{frame}
%--------------------------------------

%--------------------------------------
\begin{frame}
  \begin{figure}
    \includegraphics[scale=.8]{sw_figure1_inflation.eps}
  \end{figure}
\end{frame}
%--------------------------------------

%--------------------------------------
\begin{frame}
  \begin{figure}
    \includegraphics[scale=.8]{sw_figure4_inflation.eps}
  \end{figure}
\end{frame}
%--------------------------------------

%--------------------------------------
\begin{frame}
  \begin{figure}
    \includegraphics[scale=.8]{sw_figure1_interest.eps}
  \end{figure}
\end{frame}
%--------------------------------------

%--------------------------------------
\begin{frame}
  \begin{figure}
    \includegraphics[scale=.8]{sw_figure6.eps}
  \end{figure}
\end{frame}
%--------------------------------------

%--------------------------------------
\begin{frame}
  \begin{figure}
    \includegraphics[scale=.7]{sw_table5.eps}
  \end{figure}
\end{frame}
%--------------------------------------

%--------------------------------------
\begin{frame}
  \begin{figure}
    \includegraphics[scale=.7]{sw_table6.eps}
  \end{figure}
\end{frame}
%--------------------------------------

%--------------------------------------
\end{document}

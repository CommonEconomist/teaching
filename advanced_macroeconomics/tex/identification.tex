\documentclass{beamer}
\usetheme{}
\usecolortheme{dolphin}           
\useinnertheme{circles}
\setbeamertemplate{itemize items}[default]
\setbeamertemplate{enumerate items}[default]
\usepackage[T1]{fontenc}
\usepackage[utf8]{inputenc}
\usepackage{lmodern}
\usepackage{amsmath}
\usepackage{booktabs} 
\usepackage{graphicx}        
\usepackage{array}
\usepackage{color}
\usepackage{svg}
\makeatletter
\def\zapcolorreset{\let\reset@color\relax\ignorespaces}
\def\colorrows#1{\noalign{\aftergroup\zapcolorreset#1}\ignorespaces}
\makeatother
\graphicspath{{/home/swl/Dropbox/ucd/advanced_macro/figures/}} 
\setbeamertemplate{navigation symbols}{}
\hypersetup{colorlinks=true, linkcolor=black, citecolor=blue,urlcolor=blue} % Colours of various links
%--------------------------------------
%%%% DETAILS TITLE PAGE %%%%
%--------------------------------------
\title{Advanced macroeconomics \\ Identification}
\author{School of Economics, University College Dublin}
\date{Spring 2017}
\begin{document}

%--------------------------------------
\begin{frame}
 \titlepage
\end{frame}
%--------------------------------------

%--------------------------------------
\begin{frame}
  Macroeconometricians do four things
  \begin{enumerate}
    \item Describe and summarise macroeconomic data
    \item Make macroeconomic forecasts
    \item Quantify what we - don't- know about the macroeconomy
    \item Advise policymakers - on macroeconomic issues
  \end{enumerate}
\end{frame}
%--------------------------------------

%--------------------------------------
\begin{frame}
  Important empirical questions in macroeconomics include
  \begin{itemize}
    \item Why do some countries grow faster than others?
    \medskip
    \item What causes business cycle fluctuations?
    \item How does monetary or fiscal policy affect the economy?    
  \end{itemize}  
\medskip
Often we don't know the answers to these questions and finding out is difficult due to issues related to identification.
\end{frame}
%--------------------------------------

%--------------------------------------
\begin{frame}
  As an example, consider the FED lowering the interest rates in 2008 as a reaction to the crisis
  \begin{itemize}
    \item Lowering interest rates should encourage spending
  \end{itemize}
  \medskip
  Lowering the interest rates is a source of variation in monetary policy which can be used.
  Estimate effect on economy using OLS
  \begin{align*}
    \Delta Y_t = \alpha + \beta \Delta i_t + \epsilon_t
  \end{align*}
  Possible estimate: $\beta<0$
\end{frame}
%--------------------------------------

%--------------------------------------
\begin{frame}
 This would lead us to conclude that the reduction in interest rate correlates/causes decreases in output.
 \begin{itemize}
   \item So lowering the interest rate is not good for the economy
   \item Policy implication: increase interest rate to stimulate economic activity
 \end{itemize}
 NB - FED lowers interest rates becase some factors negatively affect the economy
 \begin{itemize}
   \item Think falling housing prices and how these affect balance sheets of banks
 \end{itemize}
 These other factors confound the effect of the change in monetary policy
 \begin{itemize}
   \item i.e. OLS regression does not capture isolated effect of interest rate
 \end{itemize}
\end{frame}
%--------------------------------------

%--------------------------------------
\begin{frame}
  Important in macroeconomics is the role of dynamics: Two important challenges
  \begin{enumerate}
    \item Difficult to identify exogenous variation in macroeconomic policy
    \item Natural experiment that can be identified are rarely those required to answer questions we're interested in. 
  \end{enumerate}
  \medskip
Result is external validity problem. 
\end{frame}
%--------------------------------------

%--------------------------------------
\begin{frame}
 Some other important issues
 \begin{enumerate}
   \item Dynamic nature of monetary and fiscal policy make it high dimensional; can have effect on both short and long run
   \item Effects of fiscal shocks depends on monetary policy (constrained by zero lower bound) and tax policy response
   \item Effect of policy depends on the economy
   \item Degree to which a policy is a surprise affects when and how strongly an economy reacts  
 \end{enumerate}
 As a result, macroeconomics tend to be structural in nature; different from empirical work seeking for causal effects.
\end{frame}
%--------------------------------------

%--------------------------------------
\begin{frame}
 Can distinguish between micro and macro moments
 \begin{itemize}
   \item Micro moments are constructed using microeconomic data on behaviour of individuals and firms
   \item Macro moments use aggregated data to identify equilibrium outcomes; informative about what type of world we live in
 \end{itemize}
\end{frame}
%--------------------------------------

%--------------------------------------
\begin{frame}
  As well as identified and unidentified moments
  \begin{itemize}
    \item Unidentified: simple statistics such as means, variances, and correlations
    \item Identified: Statistics derived from empirical strategies or causal effect estimates; designed to uncover causal effects (micro)/responses to structural shocks (macro)
  \end{itemize}
\end{frame}
%--------------------------------------

%--------------------------------------
\begin{frame}
 Finally, in terms of identification this can be at aggregate or cross-sectional level
 \begin{itemize}
   \item Most models tend to be aggregate focusing on a single economy, mainly US
   \item Cross-sectional identification is a fairly recent development
 \end{itemize}
 NB - Cross-sectional identification brings other estimation challenges
\end{frame}
%--------------------------------------

%--------------------------------------
\begin{frame}
  Example of aggregate identification for fiscal stimulus
  \begin{enumerate}
    \item Evidence coming from wars
    \item Evidence coming from VARs
  \end{enumerate}
\end{frame}
%--------------------------------------

%--------------------------------------
\begin{frame}
  

\end{frame}


\end{document}

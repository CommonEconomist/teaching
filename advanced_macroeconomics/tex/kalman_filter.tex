\documentclass{beamer}
\usetheme{}
\usecolortheme{dolphin}           
\useinnertheme{circles}
\setbeamertemplate{itemize items}[default]
\setbeamertemplate{enumerate items}[default]
\usepackage[T1]{fontenc}
\usepackage[utf8]{inputenc}
\usepackage{lmodern}
\usepackage{amsmath}
\usepackage{booktabs} 
\usepackage{graphicx}        
\usepackage{array}
\usepackage{color}
\makeatletter
\def\zapcolorreset{\let\reset@color\relax\ignorespaces}
\def\colorrows#1{\noalign{\aftergroup\zapcolorreset#1}\ignorespaces}
\makeatother
\graphicspath{{/home/swl/Dropbox/ucd/advanced_macro/figures/}} 
\setbeamertemplate{navigation symbols}{}

%--------------------------------------
\title{Kalman filter}
\author{School of Economics, University College Dublin}
\date{Spring 2018}
\begin{document}

%--------------------------------------
\begin{frame}
 \titlepage
\end{frame}
%--------------------------------------

%--------------------------------------
\begin{frame}
  \textbf{Latent variables:} Variables that are not directly observed but inferred from other variables
  \begin{itemize}
    \item Unobserved, but variable can still play important role in theoretical model
  \end{itemize}
  \medskip
  Example: Potential output
  \begin{itemize}
    \item Keynesian model inflationary pressure determined by deviation of output from potential output
    \item Consider GDP increase in last quarter but no sign of inflationary pressure: potential output increased
  \end{itemize}  
  \medskip
  Q: Can we assume that there has been a change in potential output?
\end{frame}
%--------------------------------------

%--------------------------------------
\begin{frame}
  \textbf{Signal vs. noise:} Need to have a method that extracts useful signal from data that also contains lot of noise
  \begin{itemize}
     \item w.r.t. example; potential output probably stable from quarter to quarter while there is likelt random noise fluctuations in inflation 
   \end{itemize} 
   \medskip
   One way to extract a signal is the \textbf{Kalman filter}
\end{frame}
%--------------------------------------

%--------------------------------------
\begin{frame}
  \textbf{Conditional expectations:} We want an estimate of the value of variable $X$ 
  \begin{itemize}
    \item Problem: we don't observe $X$, only $Z$ which is correlated with $X$.
  \end{itemize}
  \medskip
  Assume that $X,Z$ are jointly normally distributed
  \begin{align}
    \begin{pmatrix}      X \\Z     \end{pmatrix}
    \sim N \left ( \begin{pmatrix}      \mu_X \\ \mu_Z    \end{pmatrix},
    \begin{pmatrix}       \sigma^2_X & \sigma_{XZ} \\ \sigma_{XZ} & \sigma^2_Z     \end{pmatrix} \right )
  \end{align}  
  \medskip
  We get
  \begin{align}
    \mathbb{E}(X|Z) = \mu_X + \frac{\sigma_{XZ}}{\sigma^2_Z}(Z-\mu_Z)
  \end{align}
\end{frame}
%--------------------------------------

% Need to check this slide
%--------------------------------------
\begin{frame}
  Alternatively, define $\rho$ as correlation between $X$ and $Z$
  \begin{align}
    \rho= \frac{\sigma_{XZ}}{\sigma_X \sigma_Z}
  \end{align}
  Inserting in (2) we get
  \begin{align}
    \mathbb{E}(X|Z) = \mu_X + \rho \frac{\sigma_X}{\sigma_Z}(Z-\mu_Z)
  \end{align}
  \medskip
  Weight put on information from $Z$ depends on
  \begin{enumerate}
    \item Correlation between $X$ and $Z$ ($\rho$)
    \item The relative standard deviation ($\frac{\sigma_X}{\sigma_Z}$)
  \end{enumerate}
  If $Z$ has high standard deviation, it is a poor signal.
\end{frame}
%--------------------------------------

%--------------------------------------
\begin{frame}
  \textbf{Multivariate conditional expectations:} Can generalise from 2 to $n$ variables:
  Let $X$ be $1\; x\; n$ vector of variables and $Z$ $1\; x\; m$
  \begin{align}
    \begin{pmatrix}      X \\ Z    \end{pmatrix} \sim N \left( 
    \begin{pmatrix}      \mu_X \\ \mu_Z    \end{pmatrix},
    \begin{pmatrix}      \sum_{XX} & \sum_{XZ} \\ \sum_{XZ}' & \sum_{ZZ}    \end{pmatrix}
    \right)
  \end{align}
  \medskip 
  Expected value of $X$ conditional on $Z$ is 
  \begin{align}
    \mathbb{E}(X|Z) = \mu_X + \scriptstyle \sum_{XZ}\sum^{-1}_{ZZ} \textstyle(Z-\mu_Z)
  \end{align}
  \medskip
  This is an important formula in the Kalman filter
\end{frame}
%--------------------------------------

%--------------------------------------
\begin{frame}
  \textbf{State-space models:} Linear time-series models that mix observable and unobservable variables.
  \begin{align}
    S_t=FS_{t-1} + u_t
  \end{align}
  \textbf{State equation} - or transition equation - describes how unobservables $S_t$ evolve over time
  \begin{align}
    Z_t= HS_t +v_t
  \end{align}
  \textbf{Measurement equation}, relates set of observable variables $Z_t$ to unobservable variables $S_t$  
\end{frame}
%--------------------------------------

%--------------------------------------
\begin{frame}
  \textbf{Errors:} Both $u_t$ and $v_t$ can include either 
  \begin{enumerate}
    \item Normally distributed errors
    \item Zeros, if the described equation is an identity
  \end{enumerate}
  \begin{align}
    u_t &\sim N(0,\scriptstyle \sum^u) \\
    v_t &\sim N(0,\scriptstyle \sum^v)
  \end{align}
  \medskip
  $\sum$ might not have full matrix rank
  \begin{itemize}
    \item Or rank deficient
    \item i.e. not enough information in data to estimate equation
  \end{itemize}  
\end{frame}
%--------------------------------------

%--------------------------------------
\begin{frame}
  \textbf{Estimation:} Observed data described by
  \begin{align}
    Z_t = HS_t + v_t
  \end{align}
  Cannot observe $S_t$ but can replace it with unbiased guess based on information available at time $t$
  \begin{align}
    S_{t|t-1}
  \end{align}
  Assume that errors are normally distributed with known covariance matrix
  \begin{align}
    S_t-S_{t|t-1} \sim N (0,\scriptstyle \sum^S_{t|t-1})
  \end{align}
  Can express observed variables as
  \begin{align}
    Z_t=HS_{t|t-1}+v_t +H(S_t-S_{t|t-1})
  \end{align}
\end{frame}



%--------------------------------------
\begin{frame}
  Can estimate model using ML; variance of error term, after conditioning on $t-1$ state-variable estimate, is given by
  \begin{align}
    v_t &+ H(S_t-S_{t|t-1}) \sim N (0,\Omega_t)\\
    \Omega_t &=\scriptstyle\sum^v +H\sum^S_{t|t-1}H'
  \end{align}
  Parameters of model are given by 
  \begin{align}
    \theta=(F,H,\scriptstyle \sum^u,\sum^v)
  \end{align}
  Log-likelihood function for $Z_t$ given observables at $t-1$ is
  \begin{align}
    log\;f(Z_t|Z_{t-1},\theta)= -log\; 2\pi -log\;|\Omega_t|-\\ \nonumber \frac{1}{2}(Z_t-HS_{t|t-1})'\Omega_t^{-1}(Z_t-HS_{t|t-1})
  \end{align}  
\end{frame}
%--------------------------------------

%--------------------------------------
\begin{frame}
  Combined likelihood is given by
  \begin{itemize}
    \item Based on initial estimate of first period unobservable state $S_{1|0}$
  \end{itemize}
  \begin{align}
    f(Z_1,...,Z_T|S_{1|0},\theta)=f(Z_1|S_{1|0},\theta)\prod_{i=2}^{i=T}f(Z_i|Z_{i-1},\theta)
  \end{align}
  All we need is a method to get an unbiased guess based on information available at $t-1$: This is where the Kalman filter comes in
  \begin{itemize}
    \item Iterative method
    \item Given estimates of state variables for $t$, it uses observable data for $t+1$ to update estimates
  \end{itemize}
\end{frame}
%--------------------------------------

%--------------------------------------
\begin{frame}
  \textbf{Estimating state variables:} Formulate estimate of state variable at time $t$ given information at $t-1$
  \begin{align}
    S_t=FS_{t-1}+u_t \Rightarrow S_{t|t-1}=FS_{t-1|t-1}
  \end{align}
  At $t-1$, expected value for the observables at $t$ are
  \begin{align}
    Z_{t|t-1}=HS_{t|t-1}=HFS_{t-1|t-1}
  \end{align}
  At $t$ we observe $Z_t$; need to update estimate of state variable given information
  \begin{align}
     Z_t-HFS_{t-1|t-1}
   \end{align} 
\end{frame}
%--------------------------------------

%--------------------------------------
\begin{frame}
  Model assumptions imply
  \begin{align}
    \begin{pmatrix}      S_t \\ Z_t     \end{pmatrix} \sim N \left( 
    \begin{pmatrix}      FS_{t-1|t-1} \\HFS_{t-1|t-1}    \end{pmatrix},
    \begin{pmatrix}      \sum^S_{t|t-1} & \left( H\sum^S_{t|t-1} \right)' \\ H\sum^S_{t|t-1} & \sum^V+H\sum^S_{t|t-1}H' \end{pmatrix} \right)
  \end{align}
  \medskip
  Use conditional expectations to state that minimum variance unbiased estimate of $S_t|Z_t$ is
  \begin{align}
     \mathbb{E}(S_t|Z_t)&=S_{t|t}=FS_{t-1|t-1}+K_t(Z_t-HFS_{t-1|t-1})     
  \end{align}  
\end{frame}
%--------------------------------------

%--------------------------------------
\begin{frame}
  $K_t$ is the \textbf{Kalman gain} matrix
  \begin{align}
    K_t&= \left(H\scriptstyle \sum^S_{t|t-1} \right)' (\scriptstyle \sum^V + H\scriptstyle \sum^S_{t|t-1} \textstyle H')^{-1} 
  \end{align}
  \medskip
  Covariance matrices required to compute $K_t$ are updated by 
  \begin{align}
    \scriptstyle \sum^S_{t|t-1} \textstyle &= F\scriptstyle \sum^S_{t-1|t-1}\textstyle F'+ \scriptstyle\sum^U\\
    \scriptstyle \sum^S_{t|t} \textstyle &= (I-K_tH)\scriptstyle \sum^S_{t|t-1}
  \end{align}
\end{frame}
%--------------------------------------

%--------------------------------------
\begin{frame}
  Initialising the Kalman filter we still need
  \begin{enumerate}
    \item Initial estimate $S_{1|0}$
    \item Covariance matrix
  \end{enumerate}
  \medskip
  In many macroeconomic models $S$ can be assumed to have zero mean. For the covariance matrix we use
  \begin{align}
    \scriptstyle \sum^S_{t|t-1} = \textstyle F \scriptstyle \sum^S_{t-1|t-1} \textstyle F' + \scriptstyle \sum^u
  \end{align}
  Values of this covariance matrix will converge; for unconditional covariance matrix can use value for $\sum$ that solves
  \begin{align}
    \scriptstyle \sum=\textstyle F \scriptstyle \sum \textstyle F' + \scriptstyle \sum^u
  \end{align}
\end{frame}
%--------------------------------------

%--------------------------------------
\begin{frame}
  \textbf{Hodrick-Prescott filter}
  \begin{align}
    \sum_{t=1}^{N} [(y_t - y_t^*)^2+ \lambda(\Delta y_t^* - \Delta y_{t-1}^*)]
  \end{align}
  Consider state-space model
  \begin{align}
    y_t &= y^*_t+C_t\\
    \Delta y^*_t &= \Delta Y^*_{t-1} + \epsilon^g_t \\
    C_t &= \epsilon_t^c
  \end{align}
  Here
  \begin{align}
    Var(\epsilon^g_t) &= \sigma_g^2; Var(\epsilon_t^c) = \sigma_c^2
  \end{align}
  HP filer = Kalman filter when
  \begin{align}
    \lambda = \frac{\sigma_c^2}{\sigma_g^2} 
  \end{align}
  It is assumed that $\sigma_c^2$ is 5 percentage points and $\sigma_g^2$ one-eight percentage point: $\lambda=1600$
\end{frame}
%--------------------------------------

%--------------------------------------
\end{document}

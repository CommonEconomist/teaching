\documentclass{beamer}
\usetheme{}
\usecolortheme{dolphin}           
\useinnertheme{circles}
\setbeamertemplate{itemize items}[default]
\setbeamertemplate{enumerate items}[default]
\usepackage[T1]{fontenc}
\usepackage[utf8]{inputenc}
\usepackage{lmodern}
\usepackage{amsmath}
\usepackage{booktabs} 
\usepackage{graphicx}        
\usepackage{array}
\usepackage{color}
\makeatletter
\def\zapcolorreset{\let\reset@color\relax\ignorespaces}
\def\colorrows#1{\noalign{\aftergroup\zapcolorreset#1}\ignorespaces}
\makeatother
\graphicspath{{/home/swl/Dropbox/ucd/advanced_macro/figures/}} 
\setbeamertemplate{navigation symbols}{}
\setbeamertemplate{footline}[frame number]
%--------------------------------------
\title{Real Business Cycle}
\author{School of Economics, University College Dublin}
\date{Spring 2018}
\begin{document}

%--------------------------------------
\begin{frame}
 \titlepage
\end{frame}
%--------------------------------------

%--------------------------------------
\begin{frame}
  \textbf{Business cycles}\\
  Economies fluctuate over time: some things that need explaining
  \begin{enumerate}
    \item Volatility
    \item Comovements
    \item Persistence (autocorrelation)
    \item Effect of expectations on current decisions
  \end{enumerate}
  \medskip
  Two theoretical approached for explaining this
  \begin{enumerate}
    \item Market clearing
    \item Non-market clearing
  \end{enumerate}
\end{frame}
%--------------------------------------

%--------------------------------------
\begin{frame}
  Two competing models
  \begin{enumerate}
    \item Real Business Cycle model (RBC)
    \item New-Keynesian model (NKM)
  \end{enumerate}
  \medskip
  Similarities
  \begin{itemize}
    \item Dynamic general equilibrium
    \item Stochastic shocks
    \item Forward looking expectations
  \end{itemize}
  \medskip
  Main difference concerns information and prices
  \begin{itemize}
    \item RBC: Complete and flexible 
    \item NKM: Incomplete and sticky
  \end{itemize}
\end{frame}
%--------------------------------------

%--------------------------------------
\begin{frame}
  \textbf{Real Business Cycle model}\\
  \begin{enumerate}
    \item Take Swan-Solow growth model
    \item Insert (1) into dynamics optimisation framework
    \begin{itemize}
      \item No more constant savings rate
    \end{itemize}
    \item Add shocks to total factor productivity ($A$)
    \begin{itemize}
      \item Include uncertainty about shocks
    \end{itemize}
    \item Add leisure to account for changes in hours of work
  \end{enumerate}
\end{frame}
%--------------------------------------

%--------------------------------------
\begin{frame}
  Why real?\\
  \medskip
  Equilibrium is about
  \begin{itemize}
    \item Household preferences
    \item Technology used by firms
    \item Government policy decisions
  \end{itemize}
  \medskip
  These are \textbf{real} factors
\end{frame}
%--------------------------------------

%--------------------------------------
\begin{frame}
  Recap: \textbf{Swan-Solow model}
  \begin{align}
    Y_t &= K_t^{\alpha}(A_tN_t)^{1-\alpha}\\
    K_t &= (1+\gamma)K_{t-1} + I_t\\
    I_t &= sY_t\\
    N_t &= (1+n)N_{t-1}\\
    A_t &= (1+m)A_{t-1}\\
    Y_t &\equiv C_t + I_t
  \end{align}
  \begin{align*}
    0<\alpha<1, \gamma>0, 0<s<1
  \end{align*}
\end{frame}
%--------------------------------------

%--------------------------------------
\begin{frame}
  Fundamental mechanism of model is shocks to Total Factor Productivity (TFP)
  \begin{itemize}
    \item Recall positive correlation between GDP and productivity
  \end{itemize}
  \medskip
  Major result: Fluctuations as an equilibrium outcome
  \begin{itemize}
    \item Work harder when productivity is high: wages increase as labour becomes more productive
    \item Save more when productivity is high: interest rates increase as capital becomes more productive
  \end{itemize}
  \medskip
  Fluctuations in economy are not that bad
\end{frame}
%--------------------------------------


%--------------------------------------
\begin{frame}
  \textbf{Real Business Cycle model:} Assumes
  \begin{enumerate}
    \item Perfectly functioning competitive markets
    \item Rational expectations
  \end{enumerate}
  \medskip
  Outcomes generated by decentralized decisions of firms and households: can be replicated as solution to a social planner problem who want so maximise
 \begin{align}
  E_t \left[\sum^{\infty}_{i=0} \beta^i(U(C_{t+i})-V(N_{t+i})) \right]
 \end{align}
 $C_t$ is consumption\\
 $N_t$ hours worked\\
 $\beta$ is the household's rate of time preference 
\begin{align}
  U(C_t)-V(N_t)=\frac{C_t^{1-\eta}}{1-\eta}-\nu N_t
\end{align}
\end{frame}
%--------------------------------------

%--------------------------------------
\begin{frame}
 \textbf{Economic constraints}
\begin{align}
  Y_t &= C_t + I_t = A_tK^\alpha_{t-1}N^{1-\alpha}_t\\
  K_t &= I_t + (1-\gamma)K_{t-1}
\end{align}
  Technology process $A_t$ is usually a log-linear AR(1) process
  \begin{itemize}
    \item For simplicity assume that $A_t$ does not trend over time: economy has average growth rate of zero.
  \end{itemize}
 \begin{align}
  ln A_t= (1-\rho) ln A^* + \rho ln A_{t-1} + \epsilon_t
\end{align}
$A^*$ indicates the steady-state for technology.
\end{frame}
%--------------------------------------

%--------------------------------------
\begin{frame}
 \textbf{Solving the model}
 \begin{enumerate}
  \item Formulating the Lagrangean
  \item Finding the first order conditions (FOCs)
  \item Log-linearisation of the FOCs
  \item Finding the steady-state
\end{enumerate}
\end{frame}
%--------------------------------------

%--------------------------------------
\begin{frame}
  \textbf{Constraints}  
\begin{align}
  Y_t &= C_t + I_t = A_tK^\alpha_{t-1}N^{1-\alpha}_t\\ \nonumber
  K_t &= I_t + (1-\gamma)K_{t-1}
\end{align}
 Can combine in single equation

\begin{align}
  A_tK^\alpha_{t-1}N^{1-\alpha}_t=C_t + K_t - (1-\gamma)K_{t-1}
\end{align}
Can formulate this as a Langrangian problem

\end{frame}
%--------------------------------------

%--------------------------------------
\begin{frame} 
\begin{align}
  \mathcal{L} &= E_t \sum^{\infty}_{i=0}\beta^i[U(C_{t+i}) - V(N_{t+i})] +\\ \nonumber
  & E_t \sum^{\infty}_{i=0}\beta^i \lambda_{t+i} [A_tK^\alpha_{t+i-1}N^{1-\alpha}_t + (1-\gamma)K_{t+i-1} - C_{t+i} - K_{t+i}]
\end{align}
 The Langrangian involves picking a series of values for consumption and labour, subject to satisfying a series of constraints. 
\end{frame}
%--------------------------------------

%--------------------------------------
\begin{frame}
  \textbf{Infinity}\\ 
  Equations sums to infinity
  \begin{itemize}
    \item Entails infinite number of first-order conditions for current and expected values of $C_t, K_t,N_t$.
    \item Can be simplified by looking at when exactly the time $t$ and $t+n$ variables appear
  \end{itemize}  
\end{frame}
%--------------------------------------

%--------------------------------------
\begin{frame}
Example; Capital
\begin{align}
    \frac{\partial \mathcal{L}}{\partial K_t}
  \end{align}
  Find when $K_t$ appears.
\begin{align}
  U(C_t)-V(N_t)+ \lambda_t(A_tK^\alpha_{t-1}N^{1-\alpha}_t -C_t -K_t + (1-\gamma)K_{t-1}) \\ \nonumber
  + \beta E_t[\lambda_{t+1}(A_{t+1}K^\alpha_{t}N^{1-\alpha}_{t+1}+(1-\gamma)K_t)]
\end{align}
\end{frame}
%--------------------------------------


%--------------------------------------
\begin{frame}
   $t$ variables only appear once
  \begin{itemize}
  \item FOCs consist of differentiating the model end setting the derivatives equal to zero  
\end{itemize}
$t+n$ appear exactly as the $t$ variables: only in expectation form and multiplied by discount $\beta^n$
\begin{itemize}
  \item FOCs are identical to the $t$ variables
\end{itemize}
\end{frame}
%--------------------------------------

%--------------------------------------
\begin{frame}
\begin{align}
  Y_t = A_tK^\alpha_{t-1}N^{1-\alpha}_t 
  \end{align}
  Differentiating we get the following FOCs
\begin{align}
  \frac{\partial \mathcal{L}}{\partial C_t}&: U'(C_t)-\lambda_t=0\\
  \frac{\partial \mathcal{L}}{\partial K_t}&: -\lambda_t + \beta E_t\left[\lambda_{t+1} \left( \alpha\frac{Y_{t+1}}{K_t}+1-\gamma \right) \right] =0\\
  \frac{\partial \mathcal{L}}{\partial N_t}&: -V'(N_t) + (1-\alpha) \lambda_t \frac{Y_t}{N_t}=0\\
  \frac{\partial \mathcal{L}}{\partial \lambda_t}&: A_tK^\alpha_{t-1}N^{1-\alpha}_t - C_t - K_t + (1-\gamma)K_{t-1} =0
\end{align}
\end{frame}
%--------------------------------------

%--------------------------------------
\begin{frame}
 \textbf{Keynes-Ramsey condition}
 Now in order to make the system a bit easier to understand, it helps to define the marginal value of an additional unit of capital next year as
\begin{align}
  R_{t+1}&= \alpha \frac{Y_{t+1}}{K_t}+1-\gamma\\
  FOC &: \lambda_t=\beta E_t(\lambda_{t+1}R_{t+1})  
\end{align}

This can then be combined with the FOC for consumption to give
\begin{align}
  U'(C_t)= \beta E_t[U'(C_{t+1})R_{t+1}]
\end{align}
\begin{align}
  \frac{\partial \mathcal{L}}{\partial C_t} &= U'(C_t)-\lambda_t=0\\
  \lambda_t &= U'(C_t)
\end{align}
\end{frame}
%--------------------------------------

%--------------------------------------
\begin{frame}
  This means that
\begin{itemize}
  \item The marginal utility of consumption must equal the marginal utility of capital
  \item And the marginal utility of capital must equal the expected value of capital at $t+1$ times the return of capital times a discount factor
\end{itemize}
\end{frame}
%--------------------------------------

%--------------------------------------
\begin{frame}
 Interpretation Keynes-Ramsey condition:\\
 $\Delta$ decrease in $C_t$ will lead to utility losse
 \begin{align*}
    U'(C_t)\Delta
  \end{align*} 
  Invest to get $R_{t+1}\Delta$ next period which will be worth
 \begin{align*}
  \beta E_t[U'(C_{t+1})R_{t+1}]
 \end{align*}
 in current period's utiity
 \begin{itemize}
   \item Along an optimal path, the household must be indifferent 
 \end{itemize}
\end{frame}
%--------------------------------------



%--------------------------------------
\begin{frame}
  \textbf{CCRA Consumption and Separable Consumption-Leisure}\\
  The model uses the utility function
  \begin{align}
  U(C_t)-V(N_t)=\frac{C_t^{1-\eta}}{1-\eta}-\nu N_t
  \end{align}
  This formulation of the Constant Relative Risk Aversion (CRRA) utility from consumption and separate disutility from labour turns out to be necessary for the model to have a stable growth path solution.
\end{frame}
%--------------------------------------

%--------------------------------------
\begin{frame}
  The Keynes-Ramsey condition becomes
\begin{align}
  C^{-\eta}_t=\beta E_t(C^{-\eta}_{t+1}R_{t+1})
\end{align}

And the condition for optimal worked hours becomes
\begin{align}
  -\nu +(1-\alpha)C^{-\eta}_t \frac{Y_t}{N_t} = 0
\end{align}
\begin{align}
  \frac{Y_t}{N_t} = \frac{v}{1-\alpha}C_t^{\eta}
\end{align}
\end{frame}
%--------------------------------------


%--------------------------------------
\begin{frame}
  The RBC model can be defined by six equations
\begin{enumerate}
  \item three identities describing resource constraints
  \item one definition
  \item and two FOCs describing optimal behaviour
\end{enumerate}
Process for the technology variable is \begin{align}
  ln A_t = (1-\rho) ln A^* + \rho ln A_{t-1} + \epsilon_t
\end{align}
\end{frame}
%--------------------------------------

%--------------------------------------
\begin{frame}
\begin{align}
  Y_t &= C_t +I_t\\
  Y_t &= A_tK^{\alpha}_{t-1}N^{1-\alpha}_t\\
  K_t &= I_t+(1-\gamma)K_{t-1}\\
  R_t &= \alpha \frac{Y_t}{K_{t-1}}+1-\gamma\\
  C^{-\eta}_t &= \beta E_t(C^{-\eta}_{t+1}R_{t+1})\\
  \frac{Y_t}{N_t} &= \frac{v}{1-\alpha}C^{\eta}_t
\end{align}
\end{frame}
%--------------------------------------

%--------------------------------------
\begin{frame}
 \textbf{Taking log differences} ($\Delta\;logs$)
 \begin{align*}
   Y_t = 2X_t &\Leftrightarrow y=x\\
   Y_t = 2X_tZ_t &\Leftrightarrow y=x+z\\
   Y_t=2X_tZ_t^{-3} &\Leftrightarrow y=x-3z\\
   Y_{t+1} = X_{t+1} +Z_{t+1} &\Leftrightarrow y=x\frac{X_t}{Y_t}+z\frac{Z_t}{Y_t}\\
   Y_{t+1} = X_{t+1} + a &\Leftrightarrow y=x\frac{X_t}{Y_t}
 \end{align*}
\end{frame}
%--------------------------------------

%--------------------------------------
\begin{frame}
 \textbf{Log-linearisation}\\
 Nonlinear systems can generally not be solved analytically
 \begin{itemize}
   \item Solution can be approximated using corresponding set of linear equations: Use Taylor series
 \end{itemize}
 \medskip
 Non-linear function $F(x_t,y_t)$ can be approximated around any point $x^*_t,y^*_t$ using
 
 \begin{align}
   F(x_t,y_t) &= F(x_t^*,y^*_t)\\ \nonumber
   &+ F_x(x^*_t,y^*_t)(x_t-x^*_t) \\ \nonumber
   &+ F_y(x^*_t,y^*_t)(y_t-y^*_t)\\ \nonumber
   &+ F_{xx}(x^*_t,y^*_t)(x_t- x^*_t)^2\\ \nonumber
   &+ F_{xy}(x^*_t,y^*_t)(x_t−x^*_t) (y_t-y^*_t)\\ \nonumber
   &+ F_{yy}(x^*_t,y^*_t) (y_t-y^*_t) +...
 \end{align}
\end{frame}
%--------------------------------------

%--------------------------------------
\begin{frame}
  If the gap between ($x_t,y_t$) and ($x^*_t,y^*_t$) is small, then terms in second and higher order powers and cross-terms will all be very small and can be ignored leaving something like
\begin{align}
  F(x_t,y_t)\approx \alpha+\beta_1x_t+\beta_2y_t
\end{align}
If we linearise around point that is far away from ($x_t,y_t$), then the approximation will not be accurate.
\end{frame}
%--------------------------------------

%--------------------------------------
\begin{frame}
  \textbf{Steady-state path}\\
  DSGE models take log-linearise  variables around steady-state path
  \begin{itemize}
    \item Around this path all real variables grow at same rate
  \end{itemize}
  \medskip
  Stochastic economy will on average fluctuate around values given by steady state path
  \begin{itemize}
    \item Can get therefore an accurate approximation
    \item Provides set of linear equations in log-deviations of variables from steady-state values
    \begin{align*}
      x_t=ln\; X_t - ln\; X^*
    \end{align*}
  \end{itemize}
\end{frame}
%--------------------------------------

%--------------------------------------
\begin{frame}
 Recall: log-differences are approximately percentage deviations
 \begin{align*}
   ln\;X-ln\; Y \approx \frac{X-Y}{Y} 
 \end{align*}
 \medskip
 Approach provides
 \begin{itemize}
   \item System of variables expressed in percentage deviations from steady-state path
   \item System that can be thought of as business-cycle component of model
   \item Coefficients are elasticities (also easy with IRF)
   \item Easy to implement
 \end{itemize}
\end{frame}
%--------------------------------------
%--------------------------------------
\begin{frame}
 Write variable as
 \begin{align}
   X_t=X^*\frac{X_t}{X^*}=X^* e^{x_t} 
 \end{align}
 First-order Taylor approximation for $e^{x_t}$ given by
\begin{align}
  e^{x_t}\approx1+x_t
\end{align}
 Can write variable as
\begin{align}
  X_t \approx X^*(1+x_t)
\end{align}
\end{frame}
%--------------------------------------


%--------------------------------------
\begin{frame}
 Can set
 \begin{align}
   x_ty_t=0
 \end{align}
  \medskip
  Multiplying small deviations from steady-state will produce term close to zero anyway.
\begin{align}
  X_tY_t &\approx X^*Y^*(1+x_t)(1+y_t) \\
  &\approx X^*Y^*(1+x_t+y_t)
\end{align}
\end{frame}
%--------------------------------------



%--------------------------------------
\begin{frame}
 Example; Income
  \begin{align} 
    Y_t=C_t+I_t 
  \end{align}
  \medskip
  Rewrite
 \begin{align} 
    Y^*e^{y_t}=C^*e^{c_t}+I^*e^{i_t} 
  \end{align}
  \medskip
  Use first-order approximation
 \begin{align} 
    Y^*(1+y_t) = C^*(1+c_t) + I^*(1+i_t) 
 \end{align}
\end{frame}
%--------------------------------------

%--------------------------------------
\begin{frame}
  Steady-state terms must obey identities
  \begin{align} 
     Y^* = C^* + I^* 
  \end{align}
  \medskip
  Canceling terms on both sides
 \begin{align} 
     Y^*y_t = C^*c_t + I^*i_t 
  \end{align}
  \medskip
  Which we can write
  \begin{align} 
     y_t=\frac{C^*}{Y^*}c_t+\frac{I^*}{Y^*}i_t 
  \end{align}
\end{frame}
%--------------------------------------


%--------------------------------------
\begin{frame}
  Example: Production function
  \begin{align}
    Y_t=A_tK^{\alpha}_{t-1}N^{1-\alpha}_t 
  \end{align}
  \medskip
  Re-write in terms of steady-state and log deviations
  \begin{align} 
     Y^*e^{y_t} = (A^* e^{a_t}) (K^*)^{\alpha}e^{\alpha k_{t-1}} (N^*)^{1-\alpha}e^{(1-\alpha)n_t}
  \end{align}
  \medskip
  Steady-state must obey identities
  \begin{align} 
    Y^* = A^* (K^*)^{\alpha} (N^*)^{1-\alpha} 
  \end{align}
\end{frame}
%--------------------------------------

%--------------------------------------
\begin{frame}
  Canceling terms we get 
  \begin{align}
    e^{y_t}=e^{a_t}e^{\alpha k_{t-1}}e^{(1-\alpha)n_t} 
  \end{align}
  \medskip
  Use first-order Taylor approximation
\begin{align}
  (1+y_t)=(1+\alpha_t)(1+\alpha k_{t-1})(1+(1-\alpha)n_t)
\end{align}
\medskip
Ignore cross-products of log-deviations: simplifies to
 \begin{align} 
   y_t=a_t+\alpha k_{t-1} + (1-\alpha)n_t 
\end{align}
\end{frame}
%--------------------------------------


%--------------------------------------
\begin{frame}
  Log-linearised system
  \begin{align*}
  y_t &= \frac{C^*}{Y^*}c_t + \frac{I^*}{Y^*}i_t\\
  y_t &= a_t + \alpha k_{t-1} + (1-\alpha)n_t\\
  k_t &= \frac{I^*}{K^*}i_t + (1-\gamma)k_{t-1}\\
  n_t &= y_t-\eta c_t\\
  c_t &= E_tc_{t+1} - \frac{1}{\eta}E_t r_{t+1}\\
  r_t &= \left(\frac{\alpha}{R^*}\frac{Y^*}{K^*} \right)(y_t-k_{t-1})\\
  a_t &= \rho a_{t-1} + \epsilon_t
 \end{align*}
\end{frame}
%--------------------------------------

%--------------------------------------
\begin{frame}
 Technology is assumed to be given by
 \begin{align}
   a_t=\rho a_{t-1} + \epsilon_t
 \end{align}
  Entail no trend growth in economy
  \begin{itemize}
    \item Implies that steady-state variables are constants
  \end{itemize} 
\end{frame}
%--------------------------------------




%--------------------------------------
\begin{frame}
  \textbf{Calculating steady-state}\\
  Three steady-state variables that need to be calculated; involves terms
  \begin{align}
    \frac{C^*}{Y^*},\frac{I^*}{Y^*},\frac{I^*}{K^*},\frac{\alpha}{R^*}\frac{Y^*}{K^*}
  \end{align}
  \medskip
  Take original non-linearised RBC system: figure out what it looks like along zero growth path
  \begin{align}
    y_t=y_{t+1}=y^*
  \end{align}
  Therefore
  \begin{align}
    \frac{y_t}{y_{t+1}}=1
  \end{align}
\end{frame}
%--------------------------------------

%--------------------------------------
\begin{frame}
 Start with interest rate
 \begin{itemize}
   \item Linked to consumption behaviour via Keynes-Ramsey condition
 \end{itemize}
  \begin{align}
  C_t^{-\eta} &= \beta E_t(C_{t+1}^{-\eta}R_{t+1})\\
  1 &= \beta E_t \left( \left(\frac{C_t}{C_{t+1}} \right)^\eta R_{t+1} \right)
\end{align}
  Have no trend growth in technology
  \begin{itemize}
    \item Constant values for steady-state consumption, investment, and output
  \end{itemize}
  In steady-state we have
  \begin{align}
  C^*_t &= C^*_{t+1}=C^*\\
  R^* &= \beta^{-1}
\end{align}
\medskip
In a no-growth economy, the rate of return on capital is determined by the rate of time preference.
\end{frame}
%--------------------------------------

%--------------------------------------
\begin{frame}
   Let's look at the rate of return on capital
\begin{align}
  R_t=\alpha \frac{Y_t}{K_{t-1}}+1-\gamma
\end{align}

In steady-state we have
\begin{align}
  R^*= \beta^{-1} = \alpha \frac{Y^*}{K^*}+1-\gamma
\end{align}

So we get
\begin{align}
  \frac{Y^*}{K^*}=\frac{\beta^{-1}+\gamma-1}{\alpha}
\end{align}

Together with the steady-state interest equation this tells us that
\begin{align}
  \frac{\alpha}{R^*}\frac{Y^*}{K^*} &=\alpha \beta \left(\frac{\beta^{-1}+\gamma-1}{\alpha} \right)\\
  &= 1-\beta(1-\gamma)
\end{align}
\end{frame}
%--------------------------------------

%--------------------------------------
\begin{frame}
  Now we only have to find the ratios for 
\begin{itemize}
  \item investment-capital
  \item investment-output
\end{itemize}
Here we can use the identity
\begin{align} K_t=I_t+(1-\gamma)K_{t-1} \end{align}
\begin{align}
  K^* &= I^* + (1-\gamma)K^*\\
  K^* &= I^* + K^* - \gamma K^*\\
  I^* &= \gamma K^*\\
  \frac{I^*}{K^*} &= \gamma
\end{align}


\end{frame}
%--------------------------------------

%--------------------------------------
\begin{frame}
This identity is in steady-state and combined with the fact that $K^*_t=K^*_{t-1}=K^*$ we get
\begin{align} \frac{I^*}{K^*}=\gamma \end{align}


This can be combined with the previous steady-state ratio to give
$\frac{Y^*}{K^*}=\frac{\beta^{-1}+\gamma-1}{\alpha}$
\begin{align}
  \frac{I^*}{Y^*}=\frac{\frac{I^*}{K^*}}{\frac{Y^*}{K^*}}=\frac{\alpha \gamma}{\beta^{-1}+\gamma-1}
\end{align}

From this it follows that the consumption-output ratio must be
\begin{align}
  \frac{C^*}{Y^*}=1-\frac{\alpha \gamma}{\beta^{-1}+\gamma-1}
\end{align}
\end{frame}
%--------------------------------------

%--------------------------------------
\begin{frame}
 Final system
  \begin{align}
  y_t &= \left(1-\frac{\alpha \gamma}{\beta^{-1}+\gamma -1}\right)c_t +
  \left(\frac{\alpha \gamma}{\beta^{-1}+\gamma-1}\right)i_t\\
  y_t &= a_t +\alpha k_{t-1} + (1-\alpha)n_t\\
  k_t &= \gamma i_t + (1-\gamma)k_{t-1}\\
  n_t &= y_t-\eta c_t\\
  c_t &= E_t c_{t+1} - \frac{1}{\eta}E_t r_{t+1}\\
  r_t &= (1-\beta(1-\gamma))(y_t-k_{t-1})\\
  a_t &= \rho a_{t-1} + \epsilon_t
\end{align}
\end{frame}
%--------------------------------------

%--------------------------------------
\begin{frame}
  \textbf{Estimation}
  \begin{enumerate}
    \item Make assumption about underlying parameter values
    \item Use Binder-Pesarant algorith to get reduced-form solution
    \item Simulate model
  \end{enumerate} 
\end{frame}
%--------------------------------------


%--------------------------------------
\begin{frame}
\begin{enumerate}
  \item Perfect markets and rational expectations
  \begin{itemize}
    \item Markets are not always competitive and people are not always rational (in their economic decisions)
    \item RBC model should be seen as a benchmark against which more complicated models can be assessed. 
    \item Separate modeling of the decisions of firms and households to account for imperfect competition can be done    
  \end{itemize}

  \item Monetary and fiscal policy
  \begin{itemize}
    \item RBC models exhibit complete monetary neutrality, so there is no role at all for monetary policy, something which many people think is crucial to understanding the macroeconomy
    \item Most models build on the RBC approach introducing mechanisms that are allowed to have Keynesian effects, such as sticky prices and
    wages 
  \end{itemize}
  \item Skepticism about technology shocks
  \begin{itemize}
    \item RBC models give primacy to technology shocks as the source of economic fluctuations (all variables apart from $A_t$ are
    deterministic). But what are these shocks?
    \item Link between long-term growth and TFP
  \end{itemize}
\end{enumerate}

\end{frame}
%--------------------------------------

%--------------------------------------
\begin{frame}
  Can check the parameterizing of the the model and simulate and check the impluse response functions. 
The following graphs are based on a model with parameter values intended for the analysis of quarterly time series
\begin{align}
  \alpha &=\frac{1}{3}\\
  \beta &=0.99\\
  \gamma &=0.015\\
  \rho &= 0.95\\
  \eta &= 1
\end{align}
\end{frame}
%--------------------------------------

%--------------------------------------
\begin{frame}
  Figure shows a 200-period simulation of the model and illustrates the main feature of the RBC model, namely that it can generate business cycles that don't look too far-fetched. 
We can notice two things here
\begin{enumerate}
  \item The model roughly matches the observed fluctuations in output
  \item The model reflects the fact that investment cycles are more volatile than consumption
\end{enumerate}
\end{frame}
%--------------------------------------

%--------------------------------------
\begin{frame}
  Part of the early hype surrounding RBC models stemmed from the idea that the model contained important propagation mechanisms turning technology shocks into business cycles. 
The idea behind this is that increases in technology would lead to extra output through higher capital accumulation and by inducing people to work more. 
This entails, as suggested in early research, that in a world with identical technology level one would expect the RBC model still to generate business cycles. 
However, these propagation mechanisms are quite weak as shown in figure which illustrates that fluctuations in output follow fluctuations in technology quite closely.
\end{frame}
%--------------------------------------




%--------------------------------------
\end{document}

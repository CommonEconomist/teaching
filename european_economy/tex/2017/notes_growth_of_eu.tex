\documentclass{tufte-handout}
%------------------------------------------------
%\geometry{showframe} % display margins for debugging page layout
%------------------------------------------------
\usepackage{graphicx} % allow embedded images
  \setkeys{Gin}{width=\linewidth,totalheight=\textheight,keepaspectratio}
  \graphicspath{{/home/swl/Dropbox/ucd/eu_economics/figs/}}  % set of paths to search for images
\usepackage{amsmath}  % extended mathematics
\usepackage{booktabs} % book-quality tables
\usepackage{units}    % non-stacked fractions and better unit spacing
\usepackage{multicol} % multiple column layout facilities
\usepackage{lipsum}   % filler text
\usepackage{fancyvrb} % extended verbatim environments
  \fvset{fontsize=\normalsize}% default font size for fancy-verbatim environments

%------------------------------------------------
% Standardize command font styles and environments
\newcommand{\doccmd}[1]{\texttt{\textbackslash#1}}% command name -- adds backslash automatically
\newcommand{\docopt}[1]{\ensuremath{\langle}\textrm{\textit{#1}}\ensuremath{\rangle}}% optional command argument
\newcommand{\docarg}[1]{\textrm{\textit{#1}}}% (required) command argument
\newcommand{\docenv}[1]{\textsf{#1}}% environment name
\newcommand{\docpkg}[1]{\texttt{#1}}% package name
\newcommand{\doccls}[1]{\texttt{#1}}% document class name
\newcommand{\docclsopt}[1]{\texttt{#1}}% document class option name
\newenvironment{docspec}{\begin{quote}\noindent}{\end{quote}}% command specification environment
%------------------------------------------------

%------------------------------------------------
%%%% Details %%%%
%------------------------------------------------
\title{EU Economics: Growth of the EU}
\author{University College Dublin}
\date{Spring 2017} 

\begin{document}
\maketitle  
% GROWTH OF THE EU
\section{Growth of the EU}
% NB - Probably join this with the other lecture
For the 8 Central European Countries that joined the EU in 2004, it is in their best interest to join the Eurozone as quickly as possible since they are all small open economies which are very vulnerable to speculative attacks to their currency. 
For the small economies trade is very important. 
The trade to GDP ratio ranges from 70 to 130\%. 
In the European framework, this is likely to increase as they become more integrated. 
However, due to the level of trade openness they are very vulnerable to the whims of the international financial market which poses problems for their currencies. 

Joining the euro area implies losing autonomy over monetary policy. 
A consequence of this is that monetary policy cannot be used to stabilise country-specific shocks. 
However, there are discussion about whether monetary policy can actually help stabilise the economy, beyond a by-product of inflation-targeting. 
Shocks are hard to identify and quantify and implemented monetary policy often operates with a lag, which can be long, variable, and pretty random. 

Different countries might have different preferences with regard to the inflation rate. 


The size of their economies is very small, comparable to that of Luxembourg and added up they amount to the size of Mexico or Canada. 
They are also relatively dependent on trade. 
Here there is of course a large opportunity with access to the common market. 



EU enlargement is a key political process. 
Europe has experienced a lot of bloodshed over the centuries due to conflict and after the Second World War the continent was divided along ideological lines along the Iron Curtain. 
The process of European unification, with the aim to make the continent more secure, has progressed since the end of the Cold War to include those parts of Europe that were under control of the Soviet Union. 
%------------------------------------------------
% FIGURE: 
\begin{figure} \centering
    \includegraphics[scale=.3]{new_members.png}
    \caption{Economic size and openness of new EU member states joined in 2000s. Data: World Bank}
    \label{fig:new_members}
  \end{figure}
%------------------------------------------------
%------------------------------------------------------------------------------
\end{document}

\documentclass{tufte-handout}
%------------------------------------------------
%\geometry{showframe} % display margins for debugging page layout
%------------------------------------------------
\usepackage{graphicx} % allow embedded images
  \setkeys{Gin}{width=\linewidth,totalheight=\textheight,keepaspectratio}
  \graphicspath{{/home/swl/Dropbox/ucd/eu_economics/figs/}}  % set of paths to search for images
\usepackage{amsmath}  % extended mathematics
\usepackage{booktabs} % book-quality tables
\usepackage{units}    % non-stacked fractions and better unit spacing
\usepackage{multicol} % multiple column layout facilities
\usepackage{lipsum}   % filler text
\usepackage{fancyvrb} % extended verbatim environments
  \fvset{fontsize=\normalsize}% default font size for fancy-verbatim environments

%------------------------------------------------
% Standardize command font styles and environments
\newcommand{\doccmd}[1]{\texttt{\textbackslash#1}}% command name -- adds backslash automatically
\newcommand{\docopt}[1]{\ensuremath{\langle}\textrm{\textit{#1}}\ensuremath{\rangle}}% optional command argument
\newcommand{\docarg}[1]{\textrm{\textit{#1}}}% (required) command argument
\newcommand{\docenv}[1]{\textsf{#1}}% environment name
\newcommand{\docpkg}[1]{\texttt{#1}}% package name
\newcommand{\doccls}[1]{\texttt{#1}}% document class name
\newcommand{\docclsopt}[1]{\texttt{#1}}% document class option name
\newenvironment{docspec}{\begin{quote}\noindent}{\end{quote}}% command specification environment
%------------------------------------------------

%------------------------------------------------
%%%% Details %%%%
%------------------------------------------------
\title{European Economy: The Economic and Monetary Union}
\author{University College Dublin}
\date{Spring 2017} 

\begin{document}
\maketitle  
%------------------------------------------------------------------------------
%%%% The European Monetary Union %%%%
\section{The Economic and Monetary Union}
An important institution with regard to monetary integration is the Economic Monetary Union (EMU) which was established by the 1992 Maastricht Treaty.\footnote{The decision to form a monetary union was taken in 1988.} 
The creation of the EMU, which also lay the foundation for the Euro, progressed according to three stages\footnote{Taken from the history section on the \href{https://www.ecb.europa.eu/ecb/history/emu/html/index.en.html}{EMU webpage}.}

\begin{enumerate}
  \item Stage 1, starting in July 1990
  \begin{itemize}
    \item Complete freedom of financial transactions
    \item Increased cooperation between central banks
    \item Free use of the European Currency Unit (ECU)\footnote{The ECU is the forerunner of the Euro.}
    \item Improvement of economic convergence
  \end{itemize}
  \item Stage 2, starting in January 1994
  \begin{itemize}
    \item Establishment of the European Monetary Institute
    \item Ban on granting of central bank credit
    \item Increased cooperation on monetary policies
    \item Strengthening of economic convergence
  \end{itemize}
  \item Stage 3, starting in January 1999
  \begin{itemize}
    \item Fixing of currency exchange rates
    \item Introduction of the Euro
    \item Conduct of the single monetary policy by the European System of Central Banks
    \item Entry into effect of the intra-EU exchange rate mechanism.\footnote{This is ERM II.}
    \item Entry into force of the Stability and Growth Pact
  \end{itemize}
\end{enumerate}

The key aspects of the Maastricht Treaty that created the EMU was to 
\begin{enumerate}
  \item Guarantee price stability 
  \item Create an independent central bank. 
\end{enumerate}
You can imagine that, as with most European affairs, getting the countries to agree on the best policies to achieve price stability and more importantly for countries to give up their monetary autonomy involved a lot of difficult negotiations. 
In the end however the agreed policies were mainly created in the image of, or according to the preferences of, Germany. 
Germany had recently reunified and was the largest economy in Europe and had a very strong currency in the Deutschmark. 
As such it had to be persuaded to abandon its own currency in favour of a single common currency, joining the Euro project.\footnote{Some argue that joining the EMU was the price Germany had to pay for reunification.}
Before we discuss the EMU in more detail it is good to take a step back and look at the European Exchange Rate Mechanism which was established in preparation of the EMU.

%------------------------------------------------------------------------------
% ERM
\section{The European Exchange Rate Mechanism}
The ERM was established in order to help stabilise exchange rates across the member countries.
It provided a central exchange rate against the ECU, which in turn provided a cross-rate for all the currencies against each other. 
The hope was that the ERM would help increase trade within Europe and control inflation.\footnote{The 1970s were a period that was characterised by a lot of volatility.}
Given the use of the ECU as a benchmark currency, the ERM was based on the idea of a fixed exchange rate albeit with margins on either side of the central exchange rate. 
In practice this meant that each currency had a target zone in which the value of the currency could fluctuate relative to the ECU. 
A problem with the design of the ERM was that countries tried to retain their monetary policy autonomy which resulted in different inflation rates.\footnote{Recall that the objective was to control inflation.} 
As such the ERM had to be realigned a couple of times which lead to several speculation crises.\footnote{The market could anticipate on these realignments selling off currencies.}
Due to the destabilising nature of these realignments, countries that were prone to high inflation tried to bring down inflation rates basically by adopting the policies of the German Bundesbank, which became the ERM standard. 
This went quite well for a while, between 1987-1992 there was no realignment, but would eventually lead to a major crisis and the break down of the ERM. 
Given that the Deutschmark served as an anchor for the whole system, this meant that the Bundesbank had full autonomy; a system that was meant to be symmetric became asymmetric. 
This had two important implications
\begin{enumerate}
  \item The Bundesbank leadership wasn't very popular with the other countries\footnote{The other countries were of the opinion that monetary policy should be shared collectively and not be given to one national central bank}
  \item The 1991-1992 crisis
\end{enumerate}

Although the 6-years without realignment seemed to be a good sign, a problem was that inflation rates didn't really converge across the board. 
Countries such as France moved towards the German inflation rate, but this didn't apply to for instance Italy. 
This meant that for countries with high inflation rates their national currencies kept appreciating, resulting in a loss of competitiveness. 
Eventually three events brought down the ERM
\begin{enumerate}
  \item German reunification
  \begin{itemize}
    \item Reunification was likely to lead to inflation due to the state of East Germany
    \item The Bundesbank raised the interest rate in response to quell the risk
    \item Some countries decided not to raise the interest rate due to economic slowdown, this triggered speculative attacks on countries that had lost competitiveness. 
  \end{itemize}
  \item Denmark rejecting the Maastricht Treaty
  \begin{itemize}
    \item The Danes were the first to ratify the treaty
    \item However the population voted against it, sparking fears of contagion
  \end{itemize}
  \item French referendum
  \begin{itemize}
    \item Similar to Denmark, France held a referendum and the polls didn't look good
    \item This sparked unrest in the exchange market
    \item Specifically the Italian Lira and British Pound were targeted as they were overvalued.\footnote{The British taxpayer lost 3.3 billion Pounds.}
  \end{itemize}
\end{enumerate}

Following these events the ERM was redesigned (ERM II) basically widening the margin which leaves more room for monetary policy autonomy. 
At the moment the only member of the ERM II is Denmark.

%------------------------------------------------------------------------------
% EMU principles
\section{Main principles of the Economic and  Monetary Union}
Going back to the EMU itself, the union is based on three main principles
\begin{enumerate}
	\item Provide price stability\marginnote{Surprisingly, the treaty itself doesn't offer a definition for price stability. However, in the Eurosystem it is defined as \begin{quote}
	"the year-on-year increase in the Harmonised Index of Consumer Prices for the Eurozone of close to but below 2 per cent. Price stability is to be maintained over the medium term"	  
	\end{quote}}
	\begin{itemize}	  	
	  \item Objectives of most central banks is to keep inflation between 1.5-2 percent over the medium term (2-3 years)
	  \item In the long run monetary policy will only impact inflation, while in the short run it will also affect growth an unemployment
	\end{itemize}	

	\item Central bank independence
	\begin{itemize}
	  \item To operate effectively, the central bank should be able to do its business without outside interference
	  \item The central bank's main aim nowadays is price stability whereas others\footnote{Read government} don't mind higher inflation rates 
	\end{itemize}	

	\item Fiscal discipline
	\begin{itemize}
	  \item This is an important point as governments could create conditions that undermine the actions by the central bank 
	  \begin{itemize}
	    \item E.g. the government could lend from the banks and if the deficits become large enough, create a financial crisis. 
	    \item So the government could spend now, to create goodwill with the population, and tax later after the elections, or never.  
	  \end{itemize}
	  \item	In a monetary union the government could be waiting for fiscal transfers\marginnote{See the lecture on Optimum Currency Areas}
	  \item The treaty includes a clause on this which led to the Stability and Growth Pact \marginnote{We will discuss this in a future lecture.}  
	\end{itemize}	
\end{enumerate}

%------------------------------------------------------------------------------
% EMU and OCA
\subsection{EMU entry conditions}
Importantly, entry to the EMU, and the Eurozone, is not based on the criteria set out by the optimum currency area theory. 
Instead European leaders decided to use another set of conditions.
This allows any EU member state to join when they have shown to be able to behave according to the guiding principles of the Maastricht Treaty.\marginnote{Countries that wish to join the EU are actually obliged to join the Euro.} 
There are five entry conditions
\begin{enumerate}
	\item Inflation
	\begin{itemize}
	  \item The inflation rate should not exceed the average of the three lowest inflation rates achieved by the EU member state by 1.5 percentage points
	\end{itemize}

	\item Long-term nominal interest rate
	\begin{itemize}
	  \item The long-term interest rate should not exceed by more than 2 percentage points the average observed rate of the three lowest inflation rate countries
	  \item This is to deal with possible cheating on the first condition where countries can squeeze prices temporarily\marginnote{This follows the Fisher principle where the nominal interest rate equals the real interest rate plus the expected inflation.}
	\end{itemize}

	\item Exchange Rate Mechanism (ERM) membership
	\begin{itemize}
	  \item Countries should have taken part in the ERM at least two years without having to devalue its currency 
	\end{itemize}

	\item Budget deficit
	\begin{itemize}
	  \item Budget deficit should not exceed 3 per cent of GDP.\marginnote{This entry condition is a result of German preferences who are somewhat squeamish about inflation rates since the hyperinflation in the 1920 that hit the Weimar Republic.}
	  \item The budget deficit should correspond to public investment which is a source of economic growth
	\end{itemize}

	\item Public debt
	\begin{itemize}
	  \item Public debt should not exceed 60\% of GDP
	  \item The 60\% threshold was taken since it was the average level when the treaty was negotiated in 1991\marginnote{Note that the deficits can be altered by shifting around public spending and tax revenues.}
	  \item There is actually a clause stating that it is 60\% "or moving in that direction".\marginnote{This is due to Belgium who had a public debt that exceeded 60\%.} 
	\end{itemize}
\end{enumerate}
%------------------------------------------------------------------------------
\end{document}
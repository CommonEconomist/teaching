\documentclass{tufte-handout}
%------------------------------------------------
%\geometry{showframe} % display margins for debugging page layout
%------------------------------------------------
\usepackage{graphicx} % allow embedded images
  \setkeys{Gin}{width=\linewidth,totalheight=\textheight,keepaspectratio}
  \graphicspath{{/home/swl/Dropbox/ucd/eu_economics/figs/}}  % set of paths to search for images
\usepackage{amsmath}  % extended mathematics
\usepackage{booktabs} % book-quality tables
\usepackage{units}    % non-stacked fractions and better unit spacing
\usepackage{multicol} % multiple column layout facilities
\usepackage{lipsum}   % filler text
\usepackage{fancyvrb} % extended verbatim environments
  \fvset{fontsize=\normalsize}% default font size for fancy-verbatim environments

%------------------------------------------------
% Standardize command font styles and environments
\newcommand{\doccmd}[1]{\texttt{\textbackslash#1}}% command name -- adds backslash automatically
\newcommand{\docopt}[1]{\ensuremath{\langle}\textrm{\textit{#1}}\ensuremath{\rangle}}% optional command argument
\newcommand{\docarg}[1]{\textrm{\textit{#1}}}% (required) command argument
\newcommand{\docenv}[1]{\textsf{#1}}% environment name
\newcommand{\docpkg}[1]{\texttt{#1}}% package name
\newcommand{\doccls}[1]{\texttt{#1}}% document class name
\newcommand{\docclsopt}[1]{\texttt{#1}}% document class option name
\newenvironment{docspec}{\begin{quote}\noindent}{\end{quote}}% command specification environment
%------------------------------------------------

%------------------------------------------------
%%%% Details %%%%
%------------------------------------------------
\title{European Economy: Optimum Currency Area Theory}
\author{University College Dublin}
\date{Spring 2017} 

\begin{document}
\maketitle  
% OPTIMUM CURRENCY AREAS
\section{Optimum Currency Areas}
An important part of European integration is monetary integration, as we have seen with the introduction of the Euro. 
Within economics there is a theory that sets out a systematic way of analysis to decide whether it makes sense for a group of countries to give up their national currency in favour of a common currency. 
This theory is known as the \textbf{optimum currency area} theory.\footnote{This theory was pioneered by Mundell, but work in this areas has been done earlier by Lerner.} 

%------------------------------------------------------------------------------
% Benefits of CA
\subsection{Benefits of a Common Currency Area}
Let's start by looking at the benefits of having a common currency, or for countries to form a common currency area. 
There are a number of advantages which include
\begin{enumerate}
	\item The lowering of transaction costs
	\begin{itemize}
	  \item Due to a common currency there is no need to discuss the currency of transaction
	  \item Exchange rates will also be eliminated
	  \item As such, there is no loss of value in transaction\footnote{The European Commission looked at the loss of value within the EU by starting with 100 worth of any EU currency, exchanging it successively for all other EU currencies and found that at the end you are left with 50 worth of the currency you started with.}
	  \item Lowering of costs might lead to increases in competition
	\end{itemize}

	\item Price transparency
	\begin{itemize}
	  \item Prices are directly comparable across different regions within the currency area
	  \item Increased transparency might increase competition, which is good for consumers
	  \item	Additionally it can create trade opportunities, due to reduction of the border effect\footnote{The border effect is a substantial obstacle for international trade.}
	  \item Price transparency and the associated increases in competition will also have effects on wage setting\footnote{Countries compete with each other through exports. As the Eurocrisis has shown, adjusting wage setting can be a long and painful process.}
	\end{itemize}	

	\item Reduction in uncertainty
	\begin{itemize}
	  \item Exchange rate risk is eliminated
	  \item Removing the risk resulting from exchange range regime will be beneficial to the levels of foreign direct investment\footnote{Exchange rate fluctuations might incur losses for investors in the long term, thereby reducing FDI.}
	\end{itemize}

	\item Improvements concerning trade
	\begin{itemize}
	  \item Payments will be easier and more secure in an area that shares a currency, which will again lead to increases in competition
	  \item A common currency can also help reduce non-tariff barriers, such as reducing the monopoly power of certain firms in particular regions
	\end{itemize} 	

 	\item Quality of monetary policy
 	\begin{itemize}
 	  \item Improvements in the quality of monetary policy of course depends on the quality of the central bank
 	  \item The idea is that regions which has lower quality levels will be leveled up to a higher level with a central bank doing a better job in implementing policy
 	  \item It does involve a certain loss of national monetary policy autonomy though\footnote{Again, the Eurocrisis has shown that this might pose serious problems. We will discuss this in further detail in a future lecture.}
 	\end{itemize} 	 
\end{enumerate}

%------------------------------------------------------------------------------
\subsection{Costs of a Common Currency Area}
% Costs of CA
Although eliminating national currencies in favour for a common currency has some benefits, there are also some disadvantages associated with having a common currency area. 
These costs mainly stem from cross-regional differences on a number of different issues. 
The main concern economically speaking is the manner in which a particular region reacts to a shock.\footnote{Shocks can include everything here, from macroeconomic shocks such as a global recession to more local phenomena such as an earthquake.} 
Let's examine some of the limitations associated with a single currency area.

\begin{enumerate}
	\item Link between shocks and the exchange rate
	\begin{itemize}
	  \item A country experiencing a shock can't lower wages and prices to increase competitiveness
	  \item And there are no alternatives either
	  \item As a result, it is likely that the economy will slow down for a prolonged time\footnote{This happened to Germany at the turn of the century and the country was famously dubbed the sick man of Europe. Due to an overvalued Deutschmark the German economy had to adjust through low inflation and wage moderation. This increased competitiveness and the German economy has come out remarkably well out of the Great Recession.}
	\end{itemize}

	\item Asymmetric shocks
	\begin{itemize}
	  \item Since countries have different characteristics, they will face different type of shocks\footnote{Think about the refugee influx in Italy and Greece which are closer to Africa and the Middle East.}
	  \item In a currency union the setting of the exchange rate will affect both countries 
	  \item When one country experiences a shock the central bank has to make a decision. But this decision will likely have diverging effects across countries. The common exchange rate cannot insulate all countries. 
	\end{itemize}	
	\item Symmetric shocks with asymmetric effects
	\begin{itemize}
	  \item Countries can experience the same shock, but react differently
	  \item This can be the result of the country's socio-economic structure such as labour market regulations, the relative importance of certain sectors, such as the financial industry, external debt etc.\footnote{Think for instance about the fall out of Brexit for the other EU member states. Countries that have close economic ties to the U.K. like Ireland and Denmark are much more exposed compared to for instance Portugal and Slovenia.} 
	\end{itemize}
\end{enumerate}

%------------------------------------------------------------------------------
% OCA CRITERIA
\subsection{Criteria for a Common Currency Area}
The term optimum in optimum currency area is maybe a bit misplaced as the theory does not actually discuss optimum conditions. 
Instead, the theory just brings together the costs and benefits of sharing a currency.\footnote{Additionally, the theory doesn't even discuss which type of countries should for a currency union.} 
It only provides a set of criteria which make a currency union acceptable. 
This criteria set consists of three economic and three political criteria\footnote{These criteria are based mainly on the initial work on this topic by Mundell and later by McKinnon and Kenen. Note that the criteria are largely endogenous as they might change over time.} 
Let's examine the specific criteria. 
\begin{enumerate}
	\item Labour mobility
	\begin{itemize}
	  \item In an OCA the people should be able to move easily between regions
	  \item	This is order to deal with shocks. When the factors of production, such as labour, can move freely within the OCA these shocks can be mitigated more easily.\footnote{Other factors of production such as large machinery are of course less easy to move across countries.}
	  \item Importantly, various barrier to migration continue to exist such as cultural factors like a different language, or the skill of the migrant labourer\footnote{Migration has become an important issue in recent years for instance in context of the Brexit vote. One reason that the U.K. has attracted a disproportional large amount of migrant from other EU states, besides arguably sloppy legislature, is the fact that for instance for the average Pole English is an easier language to master than German or French. For the same reason you see that Spain has a large number of Romanian immigrants.}
	\end{itemize}

	\item Production diversification
	\begin{itemize}
	  \item Having a similar production structure and widely diversified production and exports is beneficial for a OCA
	  \item Recall that asymmetric shocks are a large problem for currency areas. The question is, how often do these shocks occur?
	  \item If these shocks are rare, the costs will just be episodic, while the profits accrue very day
	  \item Countries that will be affected most severely are those with a specialised economy\footnote{Think the reliance of Greece on a few key industries. Or the reliance of certain developing countries on the primary sector such as Nigeria or manufacturing such as Bangladesh.}
	  \item If the OCA countries all have a diversified economy, producing similar goods, this will reduce the probability of asymmetric shocks\footnote{Note that this is a very broad argument though. It is not at all clear what the level of diversification should be to reach a sort of immune state.}
	\end{itemize}

	\item Openness
	\begin{itemize}
	  \item When countries are open to trade and trade heavily with each other, they could form an OCA\footnote{Arguably trade dependence also raises political questions for instance in the debate on Scottish and Catalan independence.}
	  \item In an OCA the distinction between domestic and foreign goods is lost
	  \item Competition will equalise price of most goods when they are expressed in the same currency\footnote{Changes in the exchange rate might affect competitiveness through exports; firms might want to focus on exports at certain price levels as it is more profitable.}
	\end{itemize}

 	\item Fiscal transfers
 	\begin{itemize}
 	  \item When countries agree to compensate each other for adverse shocks, they form an OCA
 	  \item There is a moral hazard issue here
 	  \begin{itemize}
 	    \item Certain countries might be expecting these transfers to happen
 	    \item And conditional on this expectation they can be slacking
 	    \item e.g. their economy might not be diversified enough, they are too dependent on imports, or have too rigid labour markets which make adjustments long and painful\footnote{This has been an important discourse during the eurocrisis where Northern eurozone countries blame Southern eurozone countries for having lacked in fiscal discipline for instance during book periods.}
 	  \end{itemize}
 	\end{itemize}

	\item Homogeneous preferences
	\begin{itemize}
	  \item Currency union member countries must reach consensus on the best way to deal with shocks
	  \item Again this is a difficult issue in practice due to possible social and political differences across countries
	\end{itemize}

	\item Solidarity vs. nationalism
	\begin{itemize}
	  \item Common monetary policy might give rise to conflicts of national interests
	  \item In this case an OCA country needs to accept the costs in the name of a common destiny\footnote{This can be a very bitter pill to swallow.}
	  \item These costs can be accepted, as long as they are lower than the cumulative benefits
	  \item This criteria also implies that there should be a move to a political union some time in the future
	\end{itemize}	
\end{enumerate}
%------------------------------------------------------------------------------
\end{document}
